\chapter{Marco teórico}\label{chap:marco}

% Para pasar a la parte más matemática, debería comenzar con sistemas lineales y no lineales, que es lo más general. Esto me permite enmarcar filtro de kalman y sistema epidemiológicos dentro de una misma cosa.
% Luego puedo pasar a modelos epidemiológicos, para describir el Rt 
% O bien puedo pasar al filtro de Kalman. 

% Discutir idea de estado, modelo, etc. 
% El objetivo de este apartado es dar las bases teóricas que serán necesarias para comprender la metodología seguida. 
% También aquí debería incluirse una revisión bibliográfica. 

En este apartado se presentarán las bases teóricas y conceptuales que sustentan el trabajo realizado. Se presentan también los antecedentes del caso de estudio a tratar: el desarrollo de la pandemia en la Región Metropolitana de Santiago de Chile.

Este capítulo se organiza como sigue; en primer lugar, en la sección \ref{sec:epi-model}, se presentan los modelos epidemiológicos compartimentales deterministas. Se explican dos enfoques que permiten tratar con poblaciones heterogéneas: lagrangiano y euleriano, y se presenta el modelo lagrangiano propuesto en \cite{Bichara2018} que se usará en este trabajo.

%En tercer lugar, la sección \ref{sec:R0} presenta la técnica de la matriz de próxima generación, esencial en la estimación del número reproductivo básico \(\mathcal{R}_0\).
% Estructura del apartado 

La sección \ref{sec:kalman} está dedicada a la teoría del Filtro de Kalman, una técnica que permite la estimación de estados en sistemas lineales y que ha sido ampliamente extendida y utilizada, contando con aplicaciones en diversos ámbitos de ingeniería y ciencias como climatología, control de vehículos autónomos, etc. Se presentará la teoría básica del filtro y las principales extensiones que se utilizan durante el trabajo.

Contando ya con las bases teóricas, la sección \ref{sec:antecedentes} presenta los antecedentes relevantes al caso de estudio a tratar; el brote de COVID-19 en la ciudad de Santiago de Chile. Se describen características particulares de esta enfermedad y cómo influyen en el modelamiento. Se presentan características de la ciudad de Santiago que han influido el desarrollo de la pandemia y las medidas tomadas para ralentizar su avance.


%%%%%%%%%%%%%%%%%%%%%%%%%%%%%%%%%%%%%%%%%%%%
% Índice 
%%%%%%%%%%%%%%%%%%%%%%%%%%%%%%%%%%%%%%%%%%%%
% Explicación de la idea de estados, debería ser lo primero porque se va a usar mucho. 

% Sistemas lineales y no lineales, discretización, conceptos que se necesitan como base para entender todo el resto 
% \include{2-1sistemaslineales} % ahora en el anexo

% 


\section{Modelos epidemiológicos}\label{sec:epi-model}

Se presentan ahora los modelos epidemiológicos compartimentales deterministas. Primero, \ref{subsec:sir-seir} explica brevemente los modelos SIR y SEIR. A continuación, \ref{subsec:euler-lag} explica los enfoques lagrangiano y euleriano, que permiten tratar con poblaciones heterogéneas. Finalmente, \ref{modelo-clases-vs-ambientes} presenta el modelo lagrangiano propuesto en \cite{Bichara2018} que se usará en este trabajo.

\subsection{Modelos compartimentales SIR y SEIR}\label{subsec:sir-seir}

Los modelos compartimentales son ampliamente utilizados a la hora de modelar una enfermedad contagiosa. La idea tras ellos es simple: separar a la población en grupos o compartimientos de acuerdo al grado de avance de la enfermedad, y asignar reglas para pasar de un compartimiento a otro.

Un modelo muy sencillo es el \textbf{SIR}, que separa a la población en \textbf{S}usceptibles, \textbf{I}nfectados y \textbf{R}emovidos. En este trabajo se trata este modelo de manera determinista, viéndolo como el sistema de ecuaciones diferenciales ordinarias \ref{eq:sir}, pero también es posible tratarlo de manera estocástica \cite{Daley1984}.

\begin{equation}
\label{eq:sir}
\begin{aligned}
S'(t) &=  -\alpha \frac{S(t)I(t)}{N} \\
I'(t) &= \alpha \frac{S(t)I(t)}{N}- \gamma_I I(t) \\
R'(t) &= \gamma_I I(t).
\end{aligned}
\end{equation}

Los susceptibles son el grupo de la población que puede ser contagiado con la enfermedad. La enfermedad solo es propagada por los individuos infectados. Cuando un susceptible se infecta, inmediatamente comienza a contagiar a otros. Tras un tiempo los infectados dejan de contagiar y son removidos. Esto puede deberse a una recuperación de la enfermedad con inmunidad completa (no pueden volver a contagiarse), al aislamiento del contagiado o bien a su muerte debido a la enfermedad. El diagrama de flujo para pasar de un compartimiento a otro puede verse en la figura \ref{fig:sir}.

En este modelo, el total de la población dado por \(N = S + I + R\) permanece constante a lo largo del tiempo. Este modelo supone que un individuo hace en promedio \(\alpha\) contactos  ``efectivos'' o ``suficientes'' para contagiar la enfermedad por unidad de tiempo. Esto, en conjunto con la probabilidad \(S/N\) de que el individuo contactado sea susceptible, deja una cantidad de \(\alpha S I / N\) nuevos contagios por unidad de tiempo. Suponer una tasa de recuperación de la forma \(\gamma_I I\), con \(\gamma_I\) un valor positivo, implica un periodo de infección con distribución exponencial y media \(1/\gamma_I\) (más detalles en \cite{Brauer2019}).




% \ctikzfig{sir}
\begin{figure}[!h]
\centering
\begin{tikzpicture}
	\begin{pgfonlayer}{nodelayer}
		\node [style=black-edge] (0) at (-9, 0) {$S$};
		\node [style=black-edge] (4) at (-1, 0) {$R$};
		\node [style=black-edge] (5) at (-5, 0) {$I$};
		\node [style=none] (6) at (-7, 0.5) {$\alpha I/N$};
		\node [style=none] (7) at (-3, 0.5) {$\gamma_I$};
	\end{pgfonlayer}
	\begin{pgfonlayer}{edgelayer}
		\draw [style=arrow] (0) to (5);
		\draw [style=arrow, in=180, out=0] (5) to (4);
	\end{pgfonlayer}
\end{tikzpicture}

\caption{Diagrama de flujo del modelo SIR dado por las ecuaciones \ref{eq:sir}.} \label{fig:sir}
\end{figure}



Es posible complejizar este modelo añadiendo compartimientos extras. El compartimiento de \textbf{E}xpuestos permite pasar por un periodo de incubación o latencia, de \(1/\gamma_E\) en promedio, antes de comenzar a contagiar la enfermedad, como se ve en las ecuaciones \ref{eq:seir}. Esto da lugar al modelo \textbf{SEIR}, cuyo diagrama puede verse en la figura \ref{fig:seir}. Es posible agregar otro compartimientos como hospitalizados, fallecidos, etc. Desde luego, los compartimientos del modelo dependen de la enfermedad a modelar.


\begin{equation}
\label{eq:seir}
\begin{aligned}
S' &=  -\alpha \frac{SI}{N} \\
E' &= \alpha \frac{SI}{N} - \gamma_E E \\
I' &= \gamma_E E - \gamma_I I \\
R' &= \gamma_I I.
\end{aligned}
\end{equation}


% \ctikzfig{sir}
\begin{figure}[!h]
\centering
\begin{tikzpicture}
	\begin{pgfonlayer}{nodelayer}
		\node [style=black-edge] (0) at (-9, 0) {$S$};
		\node [style=black-edge] (4) at (-1, 0) {$I$};
		\node [style=black-edge] (5) at (-5, 0) {$E$};
		\node [style=none] (6) at (-7, 0.5) {$\alpha I/N$};
		\node [style=none] (7) at (-3, 0.5) {$\gamma_E$};
		\node [style=black-edge] (8) at (3, 0) {$R$};
		\node [style=none] (9) at (1, 0.5) {$\gamma_I$};
	\end{pgfonlayer}
	\begin{pgfonlayer}{edgelayer}
		\draw [style=arrow] (0) to (5);
		\draw [style=arrow, in=180, out=0] (5) to (4);
		\draw [style=arrow] (4) to (8);
	\end{pgfonlayer}
\end{tikzpicture}
\caption{Diagrama de flujo del modelo SEIR dado por las ecuaciones \ref{eq:seir}.} \label{fig:seir}
\end{figure}



\subsection{Enfoques euleriano y lagrangiano}\label{subsec:euler-lag}

Los modelos dados por las ecuaciones \ref{eq:sir} y \ref{eq:seir} suponen que la población se mezcla perfectamente, de forma que las interacciones entre individuos son homogéneas. Una forma de incluir heterogeneidad es separar a la población en grupos, en un modelo multiclase. Esto permite considerar las distintas capacidades de enfrentar y sobrevivir a la enfermedad de cada grupo, así como las diferencias en comportamiento.

La edad es un factor a considerar en varias enfermedades; niños menores de 5 años son especialmente vulnerables a enfermedades como la malaria o la tuberculosis. Adultos mayores tienen mayor probabilidad de experimentar síntomas graves al contraer influenza. En las enfermedades de transmisión sexual, las parejas tienen edades similares en muchos casos. Para más detalles de cómo considerar la edad dentro del modelamiento epidemiológico puede verse el capítulo 13 de \cite{Brauer2019}.

La ubicación espacial es otro factor a tener en cuenta; en una sociedad interconectada las enfermedades se propagan rápidamente entre un lugar y otro, y cambios en los patrones de movimiento alteran la transmisión de una enfermedad. Este tema es tratado brevemente en el capítulo 14 de \cite{Brauer2019}.

El objetivo de esta subsección es presentar dos enfoques distintivos que se observan al utilizar modelos con mezcla espacial heterogénea; euleriano y lagrangiano. Los nombres derivan de cierta similitud entre estos enfoques y las especificaciones euleriana y lagrangiana del movimiento de fluidos; en el enfoque lagrangiano, es posible ``seguir la trayectoria'' de las personas/partículas, mientras que en el euleriano, el enfoque está en las regiones y en los flujos de personas/partículas que entran o salen de ahí. Estas definiciones son bastante generales y no son excluyentes entre sí \cite{Cosner2009}.

Se supone una situación con \(n\) regiones, donde la personas puede moverse de una región a otra. Por simplicidad se supone además que infección no altera los patrones de movimiento de la población.\\


%\noindent \textbf{Enfoque euleriano}
\subsubsection*{Enfoque euleriano}

El enfoque euleriano podría describirse como ``migratorio''; las personas no pertenecen a ninguna región, sino que se mueven entre ellas. El movimiento de la población entre dos regiones cualesquiera es conocido.

Se presenta ahora un ejemplo más específico, tomado de una simplificación del modelo euleriano de \cite{Hsieh2007}. Se supone conocida una matriz \(C = (c_{ij})_{i, j = 1\dots n}\) de tasas de migración entre regiones.  \(c_{ij} \geq 0\) corresponde al número de personas por unidad de tiempo que sale desde la región \(i\) hacia la región \(j\). Por definición \(c_{ii} = 0\). 


Similar a \cite{Cosner2009}, si \(N_i(t)\) corresponde a la población presente en la región \(i\) a tiempo \(t\), entonces se puede formar un modelo migratorio considerando las personas que salen y las que entran, con las ecuaciones \ref{eq:migration}.

\begin{equation}\label{eq:migration}
N_i' = \sum_{j = 1}^n c_{ji} N_j - \sum_{j = 1}^n c_{ij} N_i.
\end{equation}

Luego, es posible considerar la propagación de la enfermedad dentro de cada región. Si \(S_{i}(t), I_i(t)\) corresponden al número de susceptibles e infectados en la región \(i\) a tiempo \(t\), y \(\alpha_{i}\) es el número promedio de contactos efectivos que hace un individuo en la región \(i\). La dinámica de los susceptibles en este ejemplo puede verse en la ecuación \ref{eq:migration-transmis}.

\begin{equation}\label{eq:migration-transmis}
S'_i = - \alpha_i\frac{S_iI_i}{N_i}.
\end{equation}

%\noindent \textbf{Enfoque lagrangiano}
\subsubsection*{Enfoque lagrangiano}

Otra forma de modelar esta situación es enfocarse en las personas; cada una habita en cierta región, aunque puede visitar otras regiones. Esto permite calcular el número de infectados que residen (no están solo de visita) en cada región. Los habitantes de una región usualmente comparten parámetros epidemiológicos (tasa de exposición, recuperación, etc). 

Los nuevos contagios se calculan a partir del número de infectados y susceptibles presentes en la región, comparado al total de personas en cada región, que pueden ser visitantes de otras regiones, o habitantes de esa región.

Un ejemplo de modelo lagrangiano se encuentra en \cite{Ruan2006}. Este define \(S_{ij}(t), I_{ij}(t), R_{ij}(t)\) como los suceptibles, infectados y recuperados de la clase \(i\) que se encuentran en la clase \(j\) a tiempo \(t\).  Esto permite calcular, definiendo \(N_{ij} = S_{ij} + I_{ij} + R_{ij}\), el total de residentes de la región \(i\) como \(N^r_i := \sum_{j = 1}^n N_{ij}\) y el total de personas presentes en la región \(i\) como \(N^p_i := \sum_{j = 1}^n N_{ji}\). La tasa de infección de la población de la región \(i\) debido a los contagios producidos en la región \(k\) se calcula simplemente como la suma de todos los contactos efectivos entre susceptibles de la clase \(i\) con algún infectado de cualquier clase, como se ve en la ecuación \ref{eq:lagran-trans}. \(\alpha_{ikj}\) corresponde a la cantidad de contactos efectivos entre la clase \(i\) y la clase \(j\) que ocurren en la región \(k\).

\begin{equation}\label{eq:lagran-trans}
S'_{ik} = -\sum_{j = 1}^n \alpha_{ikj} \frac{S_{ik}I_{jk}}{N^p_j}.
\end{equation}

Este modelo, guardando similitud con el modelo euleriano expuesto antes, supone que los residentes susceptibles de la región \(i\) viajan a otras regiones a una tasa constante \(\sigma_i\). Desde la región \(i\) van a la región \(j\) con probabilidad \(\nu_{ij}\), y vuelven a su región con tasa \(\rho_{ij}\). Esta, sin embargo, no es la única forma de trabajar con la movilidad de la población. Los artículos \cite{Cosner2009} y \cite{Bichara2015} utilizan matrices \(P = (p_{ij})_{i, j = 1 \dots n}\) de tiempos de residencia, donde \(p_{ij}\) corresponde a la fracción de su tiempo que los residentes de la clase \(i\) pasan en la región \(j\). Desde luego se cumple \(\sum_{j = 1}^n p_{ij} = 1\) para cada \(i\). Esta forma de abordar la movilidad será la más interesante para este trabajo, y se verá en más detalle en la siguiente sección.






\subsection{Modelo multiclase con dispersión virtual}\label{modelo-clases-vs-ambientes}

Si bien separar a la población en distintos grupos permite incorporar heterogeneidad en las interacciones, estos modelos añaden dificultades extra, puesto que ahora es necesario conocer de qué manera interactúan los distintos grupos. La primera dificultad reside en la noción de contacto efectivo. Si bien esta es clara en el contexto de enfermedades de transmisión sexual o enfermedades transmitidas por vectores (como la malaria y el dengue, que son transmitidas por mosquitos), el concepto es mucho más vago al referirnos a enfermedades transmitidas por contacto estrecho o por microgotas al toser, estornudar o hablar.

Otra dificultad está en la necesidad de conocer en gran detalles las interacciones entre grupos. En la sección anterior, el modelo euleriano presentado requería una matriz \(C = (c_{ij})_{i,j = 1 \dots n} \) con las tasas de migración entre cualquier par de regiones. El modelo lagrangiano, similarmente, requería las probabilidades \(\nu_{ij}\) de pasar de una región a otra, para cualquier par de regiones. Más aún, necesitaba de una matriz \(\alpha_{ikj}\) que detallara quién estaba en contacto con quién y dónde.

Este tipo de matrices de contacto es conocida como WAIFW (\textit{Who Acquires Infection From Whom} o  ¿Quién adquiere la infección de quién?) \cite{Anderson1992}. Este tipo de matrices suelen aproximarse por matrices de ``mezcla social'' o \textit{social mixing}; varios ejemplos para enfermedades transmitidas por contactos estrechos son \cite{Mossong2008}\cite{Wallinga2006}\cite{Edmunds2006}.

\cite{Bichara2015} sugiere una solución alternativa; utilizar un modelo lagrangiano con tiempos de residencia. De esta forma, el tiempo pasado en cada región y su riesgo sirve como \textit{proxy} de la probabilidad de contagios. \cite{Bichara2015} propone dos variantes de esta idea; en la primera, cada clase pertenece a una región pero pasa un tiempo en las otras regiones. Esta forma de abordar el problema es utilizada también por \cite{Cosner2009}. En la segunda variante, las clases interactúan en ambientes como hospitales, centros comerciales, escuelas, transporte público, lugares de trabajo, etc. Esto permite combinar clases y dispersión espacial en un mismo modelo. Esta variante es extendida posteriormente en \cite{Bichara2018} y es la que se usa en este trabajo.

%\textbf{euleriano} vs \textbf{lagrangeano} [Esto no he logrado entenderlo bien, las explicaciones son muy cortas].

% El primer modelo propuesto por \cite{Bichara2015} considera \(n\) clases, cada una de las cuales reside en una región (\textit{patch}). \(N_i(t), i = 1, \dots, n\) denota a la cantidad de personas que residen en la región \(i\) y \(p_{ij} \in [0,1]\) representa la fracción de tiempo que la clase \(i\) pasa en la región \(j\). En este trabajo, sin embargo, usaremos el modelo alternativo propuesto en el mismo artículo y desarrollado más ampliamente en \cite{Bichara2018}.
Se separa a la población en \(n\) clases, las cuales interactúan en \(m\) ambientes o áreas de riesgo. \(N_i(t), i = 1, \dots, n\) corresponde a la población de la clase \(i\). Suponemos que la población de la clase \(i\) pasa una proporción \(p_{ij} \in [0,1]\) de su tiempo en el ambiente \(j\), con \(\sum_{j = 1}^{m} p_{ij} = 1\) para cada \(i\). Para casos extremos, por ejemplo, podría darse que \(p_{ij} = 0\), es decir, la clase \(i\) no gasta nada de su tiempo en el ambiente \(j\), mientras que \(p_{ij} = 1\) significa que la clase \(i\) pasa todo su tiempo en el ambiente \(j\). 

Para ejemplificar se usan compartimientos SIR, sin embargo, esta metodología puede extenderse a otros compartimientos. Las ecuaciones en este caso están dadas por \ref{eq:virtual-disp}, donde \(\lambda_i\) está dado por \ref{eq:lambda}. Los términos \(b_i\) y \(d_i\) representan las tasas de natalidad y mortalidad de la clase \(i\) respectivamente.

\begin{equation}\label{eq:virtual-disp}
\begin{aligned}
%S_i(t)' &= -S_i(t) {\color{Red} \lambda_i(\vec{x}, t) } \\
S_i(t)' &=  b_i - d_i S_i - {\lambda_i(\vec{x}, t) } \\
I_i(t)' &= {\lambda_i(\vec{x}, t) } - \gamma_i I_i(t) - d_i I_i\\
R_i(t)' &= \gamma_i I_i(t)\\ 
\end{aligned}
\end{equation}

\begin{equation}\label{eq:lambda}
\lambda_i(\vec{x}, P) = \sum_{j=1}^m \beta_{j}p_{ij}S_i(t)\frac{\sum_{k=1}^{n} p_{kj} I_k}{\sum_{k=1}^{n} p_{kj}N_k}
\end{equation}

El valor \(\beta_j\) es una medida de riesgo propio del ambiente \(j\)-ésimo y depende de las condiciones ambientales y sanitarias. Representa la idea de que no es lo mismo pasar el tiempo en un parque que en el transporte público o en un bar.  la distancia que mantienen las personas ahí, la ventilación, entre otras.

Puesto que \(p_{ij}\) es la proporción del tiempo que la clase \(i\) reside en el ambiente \(j\), entonces la cantidad de personas en promedio de la clase \(i\) en el ambiente \(j\), en algún momento de tiempo \(t\), está dada por \(N_i(t) p_{ij} = S_i(t) p_{ij} + I_i(t) p_{ij} + R_i(t) p_{ij}\). La población total en el ambiente \(j\) es \(\sum_{k = 1}^m N_k p_{ij}\), de la cual \(\sum_{k = 1}^m I_k p_{ij}\) están infectados. De esta forma, el número de infectados de la clase \(i\) en el ambiente \(j\) por unidad de tiempo está dado por la ecuación \ref{eq:infectados-explicado}.

\begin{equation}\label{eq:infectados-explicado}
\underbrace{b_j}_{\text{riesgo}} \cdot \underbrace{S_i p_{ij}}_{\substack{\text{susceptibles}\\\text{de la clase } i\\ \, \text{en el ambiente } j}} \cdot \underbrace{\frac{\sum_{k = 1}^m I_k p_{ij}}{\sum_{k = 1}^m N_k p_{ij}}}_{\substack{\text{proporción}\\\text{de infectados}\\\text{en el ambiente }j}}
\end{equation}


El modelo presentado supone que la etapa de la enfermedad no afecta el comportamiento de la población. Una forma más completa de modelarlo es seguir a \cite{Bichara2018} y utilizar distintas matrices de tiempos de residencia para cada compartimiente, representando la idea de que los infectados se queden en su casa guardando cuarentena, por ejemplo.
% estado del arte. Uso en Covid 
%El modelo original ha sido usado por Baltazar Espinoza (\textit{Mobility restrictions for the control of epidemics: When do they work?}) para mostrar la ineficacia de los cordones sanitarios para reducir la cantidad total de infectados, si se usan, por ejemplo, para aislar una población de una región de bajo riesgo (una baja densidad poblacional por ejemplo) y buen acceso a salud de una de alto riesgo y acceso sanitario deficiente. 






% Modelos epidemiológicos como un caso particular de sistemas no lineales. 
%\section{Número reproductivo básico y efectivo}\label{sec:R0}

%Se sigue la explicación dada por \cite{Brauer2019}, que está basada en [Buscar autores].

El número reproductivo básico \(\mathcal{R}_0\) se define como el número esperado de casos de una enfermedad producidos por un individuo `típico' en una población completamente susceptible [Tal vez cita, es traducción directa]. Solo se consideran las infecciones secundarias (las producidas por ese individuo, y no las producidas por otros que fueron infectados por él). Para epidemias donde se pueden ignorar los efectos demográficos y donde tras enfermarse los individuos tienen inmunidad completa, la línea \(\mathcal{R}_0=1\) es la línea divisoria entre una infección que desaparece (\(\mathcal{R}_0<1\)) y una que se transforma en una epidemia (\(\mathcal{R}_0>1\)).

Una forma bastante general para calcularlo en modelos de ecuaciones diferenciales es usar la \textit{next generarion matrix}. La idea es la siguiente: separar los compartimientos en dos tipos: enfermo y sano. Los enfermos no necesariamente son contagiosos. Si hay \(n\) compartimientos de tipo enfermo, \(m\) de tipo sano y \(x \in \mathbb{R}^n, y \in \mathbb{R}^m\) son subpoblaciones de cada uno de esos compartimientos. Denotaremos \(\mathcal{F}_i\) al flujo de entrada al compartimiento \(i\)-ésimo debido a infecciones secundarias y \(\mathcal{V}_i\) al flujo de salida del compartimiento \(i\)-ésimo debido a progresión de la enfermedad, recuperación o muerte. Esto permite escribir 
\begin{equation}
\label{model-flows}
\begin{aligned}
x_i' &= \mathcal{F}_i(x,y) - \mathcal{V}_i(x,y) & i = 1, \dots, n \\ 
y_j' &= g_j(x,y) & j = 1, \dots, m
\end{aligned}
\end{equation}

Se consideran los siguientes supuestos 
\begin{itemize}
\item \(\mathcal{F}(0, y) = 0\) y \(\mathcal{V}(0, y) = 0\) para \(y \geq 0\). Esto dice que todas las nuevas infecciones son secundarias, provienen de alguien infectado dentro del sistema; no hay inmigración de individuos infectados hacia los compartimientos de enfermedad.
\item El sistema libre de enfermedad \(y' = g(0,y) \) tiene un único equilibrio que es asintóticamente estable, ie. todas las soluciones con condiciones iniciales \((0,y)\) se aproximan a un punto \((0, y_0)\) a medida que \(t \to \infty\). Esto asegura que el equilibrio libre de enfermedad es un equilibrio del sistema.
\item \(\mathcal{F}_i(x,y) \geq 0\) para \(x,y\) no negativos e \(i = 1, \dots, n\).
\item \(\mathcal{V}_i(x,y) \leq 0\) si \(x_i = 0\), \(i = 1, \dots, n\).
\item \(\sum_{i=1}^n\mathcal{V}_i(x,y) \geq 0\) si \(x,y\) son no-negativos.
\end{itemize}

Supongamos que introducimos una persona infectada al sistema libre de enfermedad. Si linealizamos en torno al punto de equilibrio libre de enfermedad \(0, y_0\) notamos que 
\[
\frac{\partial \mathcal{F}_i}{\partial y_j} (0,y_0) = \frac{\partial \mathcal{V}_i}{\partial y_j} (0,y_0) = 0
\]

Esto implica que las ecuaciones para los compartimientos de enfermedad está desacopladas del resto, por lo que se pueden escribir como 
\[
x' = (F-V)x
\]
Donde 
\[
F = \frac{\partial \mathcal{F}_i}{\partial x_j}(0, y_0) \quad
V = \frac{\partial \mathcal{V}_i}{\partial x_j}(0, y_0) 
\]

Llamamos a la matriz \(FV^{-1}\) la \textit{matrix de próxima generación} o \textit{next generation matrix}. Su radio espectral \(\rho\), es decir, el valor propio con mayor módulo, corresponde a \(\mathcal{R}_0 = \rho(FV^{-1})\).

\begin{teo}
El equilibrio libre de enfermedad de \ref{model-flows} es localmente asintóticamente estable si \(\mathcal{R}_0 < 1\) pero inestable si \(\mathcal{R}_0 > 1\).
\end{teo}





% Filtro de Kalman, una teoría específica de sistemas lineales que vamos a utilizar 
\section{Filtro de Kalman}\label{sec:kalman}

% Una vez que se tiene el modelo, si se quiere aplicar a un contexto específico, es necesario incorporar los datos para ajustar los parámetros. Dos técnicas conocidas y ampliamente utilizadas para esto son, por una parte, resolver el problema de minimización de una función de costo dada por la diferencia entre los valores dados por el modelo y las observaciones, sobre los posibles parámetros. Por otra parte, es posible seguir un enfoque bayesiano, asignando densidades a priori a cada parámetro y luego usando los datos para actualizar esas densidades mediante el método de Monte Carlo. Una posibilidad menos explorada es la asimilación de datos usando el Filtro de Kalman.

Filtro de Kalman es un método ampliamente utilizado en diversas aplicaciones de ingeniería como navegación, control de vehículos, procesamiento de señales, etc \cite{Auger2013}. Se tiene un sistema cuya evolución está dada por ciertas ecuaciones. El estado del sistema es desconocido pero se cuenta con observaciones ruidosas de este. El objetivo es estimar el estado a partir de estas observaciones; por ejemplo, estimar la posición de un vehículo cuando solo se conoce su velocidad.

Para un instante \(k\)-ésimo, hay tres formas importantes de estimación del estado \(x_k\) del sistema. El suavizado (\emph{smoothing}) intenta obtener la mejor estimación posible de \(x_k\) usando la información, es decir las observaciones, disponibles hasta un instante posterior \(k+m\). El filtraje solo dispone de información hasta la iteración \(k\)-ésima, y la predicción tiene información hasta una iteración anterior \(k-m\). Debido a la cantidad de información que conoce, el suavizado desde luego debería dar el mejor resultado de los tres, seguido por el filtraje y finalmente la predicción. 

Si bien en su versión original el filtro de Kalman estaba enfocado en un problema lineal gaussiano \cite{Kalman1960}, han surgido diversas variantes para generalizarlo a otros problemas. Algunas de las más conocidas son el Filtro de Kalman Extendido (EKF) o el Filtro ``sin olor'' \textit{Unscented} (UKF) \cite{Cai2006}, que permiten trabajar con sistemas no lineales, el Filtro de Kalman por Ensambles (EnKF) \cite{Katzfuss2016}, que es menos costoso computacionalmente y que puede usarse en problemas con gran cantidad de estados, o el Filtro H\(^\infty\) que es más robusto ante incertezas en el modelo. Varios de estos filtros tienen versiones continuas y discretas \cite{Kulikov2014}.

Esta sección está fuertemente basada en \cite{Anderson2005} y \cite{Simon2006}, y se estructura como sigue: en la subsección \ref{filtro-lineal} comenzaremos estudiando en detalle el filtro de Kalman para el caso más sencillo, en que el sistema es lineal y gaussiano. A continuación, la subsección \ref{filtro-extendido} presenta el Filtro de Kalman Extendido, una variante que permite trabajar con sistemas no lineales, y que es la que usaremos en el trabajo.

Finalmente se exponen algunas técnicas que serán de utilidad. La subsección \ref{estimacion-de-parametros} muestra una forma de estimar parámetros desconocidos de la dinámica del sistema. La subsección \ref{casiperfect} muestra una forma de incorporar restricciones a las estimaciones, y la subsección \ref{smoother} se aparta del filtraje para presentar una forma de suavizado cuando son conocidas todas las observaciones para un intervalo de tiempo.

\subsection{Filtro de Kalman lineal discreto}\label{filtro-lineal}

Para comenzar a explicar el filtro de Kalman, se decide utilizar un caso sencillo. Se tiene un sistema como el de la ecuación \ref{eq:kalman-lineal-discreto}, donde \(x_k \in \mathbb{R}^{n}\) y \(F_k, G_k \in \mathbb{R}^{n \times n}\) y \(w_k \sim \mathcal{N}(0,Q_k)\) es un vector aleatorio que representa un ruido gaussiano en la dinámica del sistema. 

\begin{equation}\label{eq:kalman-lineal-discreto}
x_{k+1} = F_k x_k + G_k w_k
\end{equation}

Los estados \(x_k\) del sistema son desconocidos, y solo es posible conocer observaciones \(y_k\) de ellos, dadas por la ecuación \ref{eq:kalman-lineal-discreto-obs}, donde \(y_k \in \mathbb{R}^{m}, H_k \in \mathbb{R}^{m \times n}\) y \( v_k \sim \mathcal{N}(0, R_k)\) representa un ruido gaussiano en las observaciones.

\begin{equation}\label{eq:kalman-lineal-discreto-obs}
y_k = H_k x_k  + v_k
\end{equation}


Para una condición inicial gaussiana \(x_0 \sim \mathcal{N}(\hat{x}_0, P_0)\), la linealidad del sistema y el hecho de que el ruido es gaussiano, implica que cada uno de los \(x_k\) serán también gaussianos. Se busca estimar \(x_k\) a partir de las observaciones \(y_k\). Más específicamente, en términos bayesianos, se busca la distribución \textit{a posteriori} de \(x_k\) dado un conjunto de observaciones \(y_{1:l} := \{y_1, \dots, y_l\}\). Denotaremos a esta distribución \(x_{k,l}:= x_k | y_{1:l}\).

Existen tres formas de estimación, que dependen del valor de \(l\); en primer lugar, si \(l < k\), se está intentando estimar \(x_k\) con información hasta un instante \(l\) anterior a \(k\). A esta forma se le llama predicción o \textit{forecasting}. En segundo lugar, si \(l>k\), entonces se está estimando \(x_k\) con información hasta un instante \(l\) posterior a \(k\). Se llamará a esto suavizado o \textit{smoothing}. Finalmente, si \(l = k\) se llamará filtraje o \textit{filtering}. Es esta última forma la que nos presenta mayor interés y que facilita la explicación, por lo que es la forma que será utilizada.

Se sigue el siguiente procedimiento. Se comienza suponiendo que la distribución a posteriori de \(x_{k-1}\) con observaciones hasta el instante \(k-1\) es conocida, \(x_{k-1, k-1} \sim \mathcal{N}(\hat{x}_{k-1, k-1}, P_{k-1, k-1})\); se busca la distribución \(x_{k, k}\). Esto se logra mediante una iteración del filtro de Kalman, la cual se divide en dos etapas; una de predicción, que usa la dinámica para obtener \(x_{k+1, k}\), y una de análisis, que incorpora \(y_k\) para obtener \(x_{k, k}\).

La propiedad \ref{prop:conj-gauss-cond}, cuya demostración puede leerse en \cite{Anderson2005} y que ocupa el Teorema de Bayes, será útil a la hora de hacer los cálculos.


\begin{prop}
\label{prop:conj-gauss-cond}
Sean \(X, Y\) vectores aleatorios conjuntamente
gaussianos, ie. tal que

\[
Z := \begin{pmatrix}
X\\
Y
\end{pmatrix} \sim \mathcal{N}\left(
\begin{pmatrix}
\bar{x} \\
\bar{y}
\end{pmatrix}, 
\begin{pmatrix}
\Sigma_{xx} & \Sigma_{xy} \\
\Sigma_{yx} & \Sigma_{yy}
\end{pmatrix}
\right)
\] Entonces \[
(X|Y=y) \sim \mathcal{N} \left( \bar{x} + \Sigma_{xy}\Sigma_{yy}^{-1}(y-\bar{y}),
\Sigma_{xx} - \Sigma_{xy}\Sigma_{yy}^{-1}\Sigma_{yx}
 \right)
\]
\end{prop}


\begin{enumerate}
\def\labelenumi{\arabic{enumi}.}
\item
  En el paso de predicción se hace una suposición \emph{a priori} de dónde estará el estado real en la próxima iteración a partir de la dinámica y de la estimación actual, sin utilizar la observación \(y_k\). Usando que \(x_{k-1,k-1} \sim \mathcal{N}(\hat{x}_{k-1,k-1}, P_{k-1,k-1})\), y la ecuación \ref{eq:kalman-lineal-discreto}, no es difícil ver que \(x_{k,k-1} \sim \mathcal{N}(\hat{x}_{k,k-1}, P_{k,k-1})\), con 
    \begin{equation}
    \begin{aligned}
    \hat{x}_{k,k-1} &= F_{k-1} \hat{x}_{k-1,k-1} \\
    %P_{k,n-1} &= M_k P_{k-1,n-1} M_k' + F_k Q_k F_k'
    P_{k,k-1} &= F_{k-1} P_{k-1,k-1} F_{k-1}^{T} + G_k Q_k G_k^T
    \end{aligned}
    \end{equation}
\item
  En el paso de análisis, se incorpora la observación \(y_k\) para obtener una estimación \emph{a posteriori}. Para esto basta con usar la proposición \ref{prop:conj-gauss-cond}, calculando la distribución conjunta de \(x_k\) e \(y_k\) dado \(y_{1:k}\), para obtener la expresión \ref{eq:conjunta}. La propiedad \ref{prop:conj-gauss-cond} nos permite decir que \(
    x_{k,k} \sim
     \mathcal{N} \left( \hat{x}_{k,k}, P_{k,k} \right)
    \), con \(\hat{x}_{k,k}\) y \(P_{k,k}\) definidos en las ecuaciones \ref{eq:terminos-raros}.
  
    \begin{equation}\label{eq:conjunta}
    \begin{pmatrix}
    x_k \\
    y_k
    \end{pmatrix} \Big| y_{1:k} \sim \mathcal{N}\left(
    \begin{pmatrix}
    \hat{x}_{k, k-1} \\
    H_k \hat{x}_{k,k-1}
    \end{pmatrix},
    \begin{pmatrix}
    P_{k,k-1} & P_{k,k-1} H^T_{k}\\
    H_{k}P_{k,k-1}  & H_{k}P_{k,k-1}H_{k}^T + R_k
    \end{pmatrix}  
    \right)
    \end{equation}

    \begin{equation}\label{eq:terminos-raros}
    \begin{aligned}
    K_k &= P_{k,k-1} H'_{k}(H_{k}P_{k,k-1}H_{k}' + R_k)^{-1}\\
    \hat{x}_{k,k} &= \hat{x}_{k, k-1} + K_k(y_k-H_k \hat{x}_{k,k-1}) \\
    P_{k,k} &= P_{k,k-1}- K_k H_{k}P_{k,k-1} \\
    \end{aligned}
    \end{equation}
\end{enumerate}
Estos dos pasos de predicción y análisis entregan una regla que permite iterar; basta con definir la condición inicial como \(\hat{x}_{0,-1} := x_0\), \(P_{0,-1} := P_0\). El Teorema \ref{teo:min-var}, cuya demostración está en \cite{Anderson2005}, asegura que los valores \(\hat{x}_{k,k}\) son estimadores de mínima varianza para \(x_{k,k}\).

\begin{teo}
\label{teo:min-var}
Sean \(X, Y\) vectores aleatorios conjuntamente gaussianos. Sea también \(\hat{x} := \mathbb{E}(X|Y=y)\). Entonces \(\hat{x}\) es un
\emph{estimador de mínima varianza}, es decir, satisface

\[
\mathbb{E}(\| X - \hat{x}\|^2 | Y = y) \leq \mathbb{E}(\| X - z\|^2 | Y = y) \quad \forall z
\]
\end{teo}


\begin{mdframed}[style=mystyle,frametitle=Filtro de Kalman Lineal Discreto]

Para el caso lineal discreto como el presentado por las ecuaciones \ref{eq:system-ecs}, el filtro de Kalman se define como \ref{eq:kalman-final}, donde \(\hat{x}_{0,-1}:= x_0\) y \(P_{0,-1}:= P_0\). Los valores \(\hat{x}_{k,k}\) son estimadores de mínima varianza de \(x_{n,n}\). 

\begin{equation}\label{eq:system-ecs}
\begin{aligned}
x_{k+1} &= M_k x_k + F_k w_k \\ 
y_k &= H_k x_k + v_k
\end{aligned}
\end{equation}


\begin{equation}\label{eq:kalman-final}
\begin{aligned}
\hat{x}_{k,k-1} &= F_{k-1} \hat{x}_{k-1,k-1} \\
P_{k,k-1} &= F_{k-1} P_{k-1,k-1} F_{k-1}' + G_{k-1} Q_{k-1} G_{k-1}'\\
K_k &= P_{k,k-1} H'_{k}(H_{k}P_{k,k-1}H_{k}' + R_k)^{-1}\\
\hat{x}_{k,k} &= \hat{x}_{k, k-1} + K_k(y_k-H_k \hat{x}_{k,k-1}) \\
P_{k,k} &= P_{k,k-1}- K_k H_{k}P_{k,k-1} \\
\end{aligned}
\end{equation}

\end{mdframed}


% {[}Agregar recuadro con la variante que incorpora un control conocido{]}

% Similarmente, para el modelo 
% \[
% \begin{aligned}
% x_{k+1} &= M_k x_k + B_k u_k + F_k N_k \\ 
% y_k &= H_k x_k + v_k
% \end{aligned}
% \]

% Las ecuaciones correspondientes son 

% {[}Agregar un ejemplo y algún gráfico, para hacerse una idea de cómo funciona{]}

\subsection{Filtro de Kalman Extendido (EKF)}\label{filtro-extendido}

% agregar más intro

La idea de esta sección es extender los métodos a sistemas no lineales. Trabajaremos ahora con un modelo de la forma 

\begin{equation}
\begin{aligned}
x_{k+1} &= f_k(x_k) + g_k(x_k) w_k \\
y_{k} &= h(x_k) + v_k
\end{aligned}
\end{equation}

donde las funciones \(f_k, g_k, h_k\) son posiblemente no lineales. 

Denotaremos 
\begin{equation}
\begin{aligned}
F_k := \left. \frac{\partial f_k(x)}{\partial x} \right|_{x = \hat{x}_{k,n}} & H_k := \left. \frac{\partial h_k(x)}{\partial x} \right|_{x = \hat{x}_{k,n}} & G_k := g_k(\hat{x}_{k,n})
\end{aligned}
\end{equation}
Usando series de Taylor, podemos expandir 

\[
\begin{aligned}
f_k(x_k) &= f_k(\hat{x}_{k,n}) + F_k(x_k - \hat{x}_{k,n}) + \dots \\
g_k(x_k) &= g_k(\hat{x}_{k,n}) + \dots = G_k + \dots \\
h_k(x_k) &= f_k(\hat{x}_{k,n}) + H_k(x_k - \hat{x}_{k,n}) + \dots \\
\end{aligned}
\]

Despreciando los términos de mayor orden, y suponiendo conocidos \(\hat{x}_{k,n}\) y \(\hat{x}_{k,n-1}\) podemos aproximar el sistema como 

\begin{equation}
\begin{aligned}
x_{k+1} &= F_k x_k + G_k w_k + \overbrace{f_k(\hat{x}_{k,n}) - F_k \hat{x}_{k,n}}^{u_k} \\
y_{k} &= H_k x_k + v_k + \underbrace{h_k(\hat{x}_{k,n-1}) - H_k \hat{x}_{k,n-1}}_{z_k}
\end{aligned}
\end{equation}


Ahora podemos usar simplemente las ecuaciones del filtro de Kalman lineal, salvo en el paso de \textit{forecast}, donde conservaremos la dinámica no lineal. Definiendo \(\hat{x}_{0,-1}:= \bar{x}_0\) y \(P_{0,-1}:= P_0\), las ecuaciones del EKF son:

\begin{equation}\label{eq:kalman-extended}
\begin{aligned}
\hat{x}_{k,n-1} &= f_{k-1}( \hat{x}_{k-1,n-1}) \\
P_{k,n-1} &= F_{k-1} P_{k-1,n-1} F_{k-1}' + G_k Q_k G_k'\\
K_k &= P_{k,n-1} H'_{k}(H_{k}P_{k,n-1}H_{k}' + R_k)^{-1}\\
%z_k &= h_k(\hat{x}_{k,n-1}) - H_k \hat{x}_{k,n-1} \\
\hat{x}_{k,n} &= \hat{x}_{k, n-1} + K_k(y_k-h_k(\hat{x}_{k,n-1})) \\
P_{k,n} &= P_{k,n-1}- K_k H_{k}P_{k,n-1} \\
\end{aligned}
\end{equation}





% \subsection{Filtro continuo-discreto}\label{filtro-continuo-discreto}

% \cite{Kulikov2017} Queremos seguir extendiendo la idea anterior. Supondremos ahora que nuestro sistema es continuo y está definida por una ecuación diferencial estocástica de la forma 

% \[
% dx(t) = f(t, x(t))dt + g(t)dw(t)
% \]

% donde \(w(t)\) es un proceso browniano con matriz de difusión \(Q(t)\). Las observaciones son discretas y llegan cada intervalos de largo \(\delta := t_k - t_{k-1}\). Llamaremos a \(\delta\) \textit{tiempo de muestreo}. 

% Para poder usar las variantes anteriores es necesario discretizar la ecuación. Hay dos opciones; es posible linealizar primero y luego discretizar, dando lugar al filtro extendido \textit{continuo-discreto} (CD-EKF), o discretizar y luego linealizar, a lo que se llama filtro \textit{discreto-discreto}. Nos interesa el primer caso. 

% Obtendremos la siguiente Ecuación Diferencial Ordinaria (EDO) 
% \[
% \begin{aligned}
% \frac{d\hat{x}}{dt} &= f(t, \hat{x}) \\ 
% \frac{dP}{dt} &= \frac{\partial f(t, \hat{x})}{\partial \hat{x}} P + P \left( \frac{\partial f(t, \hat{x})}{\partial \hat{x}} \right)^{T} + g(t)Q(t)g(t)^{T} \\ 
% \end{aligned}
% \]
% la que tendremos que resolver en el intervalo \([t_{k-1}, t_{k}]\) con las condiciones iniciales \(\hat{x}(t_{k-1}) = \hat{x}_{k-1, n-1}, P(t_{k-1}) = P_{k-1, n-1}\), para luego definir \(\hat{x}_{k,n-1} := \hat{x}(t_k), P_{k,n-1} := P(t_k)\). El paso de análisis permanece sin cambios. 

% Notamos que este enfoque tiene dos problemas importantes; la ecuación debe ser tratada numéricamente, lo que introduce un error de discretización adicional. La segunda se relaciona con la semipositividad de la matriz \(P(t_k)\), la cual no está garantizada.

% \subsection{Filtro por ensambles}\label{filtro-por-ensambles}

% Esta sección está basada en \cite{Katzfuss2016}.

% El filtro de Kalman por ensambles (\textbf{EnKF} por su nombre en inglés \emph{Ensemble Kalman Filter}), es una versión aproximada del filtro de Kalman, donde la distrubución del estado se representa con una muestra o
% ``ensamble'' de esa distribución .

% Más específicamente, suponemos que \(\hat{x}^{(1)}_{k,n}, \dots, \hat{x}^{(N)}_{k,n}\) es una muestra de la distribución filtrada en tiempo \(k\), es decir, que \(\hat{x}^{(i)}_{k,n} \sim \mathcal{N}(\hat{x}_{k,n}, P_{k,n})\).

% Al igual que con el filtro de Kalman, una iteración de EnKF también se hará en un paso de \emph{forecast} y un paso de análisis.

% Durante el \emph{forecast}, nuestra distribución a priori será

% \[
% \hat{x}^{(i)}_{k+1,n} = f_k(\hat{x}^{(i)}_{k,n}, u_k, w^{(i)}_k)
% \]

% con \(w^{(i)}_k \sim \mathcal{N}(0,Q_k)\). Para el caso lineal en que \(f_k(x, u, w) = M_k x + B_k u + w\) se puede verificar que
% \(\hat{x}^{(i)}_{k+1,n} \sim \mathcal{N}(\mu, \Sigma)\) (COMPLETAR).

\subsection{Estimación de
parámetros}\label{estimacion-de-parametros}

Hasta ahora hemos trabajado con el supuesto de que las únicas cantidades desconocidas eran los vectores \(x_k\). Sin embargo, la mayoría de las veces tendremos además parámetros desconocidos \(p\), los cuales también necesitan ser estimados. \\

\noindent\textbf{Estado aumentado}

Este apartado está basado en la sección 13.4 de \cite{Simon2006}. Supongamos que tenemos un sistema donde las matrices dependen de forma no lineal de cierto parámetro desconocido \(p\). 

\[
\begin{aligned}
x_{k+1} &= F_k(p)(x_k) + G_k(p)u_k + L_k(p)w_k \\
y_{k} &= H_k x_k + v_k \\ 
\end{aligned}
\]

Por simplicidad agregó dependencia de \(p\) a las mediciones, pero es un caso que puede extenderse de manera sencilla a partir de este. Para estimar \(p\) se define un estado aumentado \(x'\):

\[
x'_k = \begin{bmatrix}x_k \\ p_k \end{bmatrix}
\]
Supondremos que \(p\) es constante y usaremos la dinámica \(p_{k+1} = p_k + w_{pn}\). El ruido \(w_{pn}\) artificial permitirá al filtro cambiar su estimación de \(p_k\). 

Finalmente, nuestro sistema aumentado es 

\[
\begin{aligned}
x'_{k+1} &= \begin{bmatrix} F_k(p_k)(x_k) + G_k(p_k)u_k + L_k(p_k)w_k \\ p_k + w_{pn} \end{bmatrix} \\
&= f(x'_{k}, u_k, w_k, w_{kp}) \\
y_{k} &= \begin{bmatrix} H_k & 0 \end{bmatrix} \begin{bmatrix}x_k \\ p_k \end{bmatrix} + v_k \\ 
\end{aligned}
\]

Notamos que \(f(x'_{k}, u_k, w_k, w_{kp}) \) es una función no lineal en el estado aumentado \(x'_k\), por lo que podemos usar cualquier filtro no lineal para estimar \(x'_k\).\\

% Mostrar que es posible que el filtro converja a valores erróneos, ver cómo eso depende de los errores que se le dan al filtro.

\subsection{Suavizado o \textit{smoothing} } \label{smoother}

% Hay tres variantes de smoothing, pero a mí solo me interesa una. Las otras creo que solo las mencionaré 

% Un recordatorio de la notación 
Recordamos que \(\hat{x}_{k,m}\) corresponde a la estimación de \(x_k\) utilizando las observaciones hasta tiempo \(m\); \(\left. y \right|_{1:m}\). Hasta ahora sólo no hemos concentrado en el proceso de filtraje y solo hemos calculamos \(\hat{x}_{k,n}\) o \(\hat{x}_{k, n-1}\), es decir, nuestra estimación del estado \(x_k\) usa las observaciones solo hasta el estado \(n\), \(\left. y \right|_{1:k}\), o hasta el estado \(n-1\),   \(\left. y \right|_{1:k-1}\). En ciertas situaciones, sin embargo, es interesante mejorar una estimación usando observaciones obtenidas posteriormente, o de manera más técnica, calcular \(\hat{x}_{k, n+1}, \hat{x}_{k, n+2}, \hat{x}_{k, n+3}, \dots\).El proceso de obtener estas estimaciones usando datos de tiempos posteriores es llamado \textit{smoothing} o suavizado. El nombre se debe a que, al tener más información disponible, es posible generar estimaciones con menos ruido, más ``suaves''.

Hay tres escenarios comunes donde surge esta necesidad. El primero de ellos es el \textbf{suavizado de punto fijo}, donde nos interesa la estimación de un estado en un tiempo fijo, digamos \(j\), y el número de observaciones disponibles aumenta continuamente. Un ejemplo de esto podría ser ... [insertar ejemplo o paper donde se use]. En este caso estamos intentando estimar \(\hat{x}_{j, j+1}, \hat{x}_{j, j+2}, \hat{x}_{j, j+3}, \dots\). 

Un segundo escenario es el \textbf{suavizado con retardo fijo}. En este caso, para cada estado \(k\) se tienen \(N\) observaciones posteriores, donde \(N\) es un número fijo. Se intenta estimar \(\hat{x}_{k, k+N}\) para \(k = 1, 2, \dots, \). Un ejemplo de esto podría ser... [insertar ejemplo o paper donde se use]. 

El último escenario es \textbf{suavizado con intervalo fijo}. En este caso, tenemos un intervalo de \(M\) mediciones \(\left. y \right|_{1:M}\)y queremos obtener la mejor estimación posible de todos los estados hasta tiempo \(M\). Es decir, buscamos \(\hat{x}_{0,M}, \hat{x}_{1,M}, \hat{x}_{2,M}, \dots, \hat{x}_{M,M}\). Insertar ejemplo.

De estos tres escenarios, el suavizado con intervalo fijo será el de mayor interés para este trabajo. Debido a la extensión de las demostraciones, y a la amplia variedad de versiones disponibles, solo se dejaremos las fórmulas correspondientes a la versión que usaremos. El RTS  para más detalles acerca de cómo deducirlas, o para los otros tipos de suavizado, ver \cite{Simon2006}.  \\ 

\noindent \textbf{Suavizado con intervalo fijo: el RTS \textit{smoother}} 


% Presentar tal vez un poco la idea de forward backward. Al menos creo que tendré que mostrar la notación. 

El suavizado de intervalo fijo suele realizarse en dos pasadas; una ``hacia adelante'' en tiempo, (a esta parte se le llama \textit{forward}) usando el Filtro de Kalman estándar, con el que se obtienen las estimaciones usando los datos pasados. Luego, se suaviza usando ``hacia atrás'' en el tiempo (\textit{backward}).

Se han desarrollado varias formas de suavizado de intervalo fijo. Una de las más comunas es la de Rauch, Tung y Striebel [insertar cita], conocida como \textit{RTS smoother}. Es una forma eficiente ya que 

\textit{The RTS smoother is more computationally efficient than the smoother presented in the previous section because we do not need to directly compute the backward estimate or covariance in order to get the smoothed estimate and covariance. }
% Insertar un cuadrito
El sistema está dado por 

\[
\begin{aligned}
x_k &= F_{k-1}x_{k-1} + G_{k-1}u_{k-1} + w_{k-1} \\ 
y_k &= H_k x_k + v_k \\ 
w_k &\sim  \mathcal{N}(0, Q_k)\\
v_k &\sim  \mathcal{N}(0, R_k)
\end{aligned}
\] 

\begin{enumerate}
    \item Inicializar el filtro \textit{forward}.
    
    \[
    \begin{aligned}
    \hat{x}_{f0} &= \mathbb{E}(x_0) \\
    P^{+}_{f0} &= \mathbb{E}\left[(x_0 - \hat{x}_{f0})(x_0 - \hat{x}_{f0})^{T}\right] 
    \end{aligned}
    \]
    
    \item Para \(k = 1, \dots, N\) (donde \(N\) es el tiempo final), ejecutar el filtro \textit{forward}, que es simplemente el filtro de Kalman estándar. 
    
    \[
    \begin{aligned}
    P^{-}_{fk} &= F_{k-1}P^{+}_{f,k-1}F_{k-1}^{T} + Q_{k-1} \\ 
    K_{fk} &= P^{-}_{fk} H_k^{T} (H_k P^{-}_{fk}H_k^{T} + R_k)^{-1}  \\
    \hat{x}^{-}_{fk} &= F_{k-1}\hat{x}^{+}_{f,k-1} + G_{k-1}u_{k-1} \\
    \hat{x}^{+}_{fk} &= \hat{x}^{-}_{fk} + K_{fk} \left(y_k - H_k \hat{x}^{-}_{fk} \right)\\ 
    P^{+}_{fk} &= (I - K_{fk}H_k) P^{-}_{fk} (I-K_{fk}H_k)^{T} + K_{fk}R_{k}K_{fk}^{T} \\
    &= \left[ (P^{-}_{fk})^{-1} + H_k^{T} R_{k}^{-1}H_k\right]^{-1} \\ 
    &= (I - K_{fk}H_k)P^{-}_{fk}
    \end{aligned}
    \]
    
    \item Inicializar el \textit{RTS smooter} 
    \[
    \begin{aligned}
    \hat{x}_{k} &= \hat{x}^{+}_{fN} \\
    P_k &= P^{+}_{fN}
    \end{aligned}
    \]
    \item Para \(k = N-1, \dots, 1, 0\) ejecutar las ecuaciones del \textit{RTS smoother} 
    \[
    \begin{aligned}
    \mathcal{I}^{-}_{f, k+1} &= \left( P^{-}_{f, k+1}\right)^{-1} \\
    K_k &= P^{+}_{fk} F_{k}^{T} \mathcal{I}^{-}_{f, k+1} \\
    P_k &= P^{+}_{fN} - K_k (P^{-}_{f, k+1} - P_{k+1})K_k^{T} \\ 
    \hat{x}_k &= \hat{x}^{+}_{fk} + K_k(\hat{x}_{k+1} - \hat{x}^{-}_{f,k+1})
    \end{aligned}
    \]
\end{enumerate}



\subsection{Observaciones (casi) perfectas} \label{casiperfect}

El survey \cite{Simon2010} presenta varias alternativas para tratar con restricciones, y se decide usar la técnica de \textbf{Observaciones (casi) perfectas} para mantener constante el total de la población. Para la positividad de las variables, se decide simplemente aplicar la función \(\max(0, x)\) al estado después de los pasos de \textit{forecasting} y análisis.

La idea de Observaciones (casi) perfectas es la siguiente: en un sistema de la forma \(x_{k+1} = f_k(x_k) + w_k\), observado mediante una ecuación \(y_k = Hx_k + v_k \), se busca imponer una restricción \(|Dx - d| \leq \epsilon\). Esto se consigue ampliando la matriz de observaciones como muestra la ecuación \ref{eq:perfect-obs}. Si se busca una restricción fuerte con \(\epsilon = 0\), la técnica es llamada Observaciones perfectas, y si se busca una restricción suave con \(\epsilon > 0\), recibe el nombre de observaciones casi perfectas.


\begin{equation}\label{eq:perfect-obs}
\begin{bmatrix}
y_k \\
d 
\end{bmatrix} = 
\begin{bmatrix}
H \\
D
\end{bmatrix} x_k
+
\begin{bmatrix}
v_k \\
\epsilon  
\end{bmatrix}
\end{equation}




% Mostrar que es posible que el filtro converja a valores erróneos, ver cómo eso depende de los errores que se le dan al filtro.

%%%%%%%%%%%%%%%%%%%%%%%%%%%%%%%%%%%%%%%%%%%%%%

% \subsection{Recursos adicionales}\label{recursos-adicionales}

% \begin{itemize}
% \itemsep1pt\parskip0pt\parsep0pt
% \item
%   \href{https://www.kalmanfilter.net/default.aspx}{KalmanFilter.NET}
%   contiene explicaciones y tutoriales.
% \end{itemize}

% Un libro más matemático es

% \emph{Optimal Filtering} de Brian Anderson y John Moore.

% \begin{itemize}
% \itemsep1pt\parskip0pt\parsep0pt
% \item
%   Katzfussa, Stroudb, Wiklec, ``Understanding the Ensemble Kalman
%   Filter''
% \end{itemize}


% Antecedentes específicos de la ciudad de santiago y del covid que necesitan tenerse en consideración
\section{Antecedentes: El caso del COVID-19} \label{sec:antecedentes}

El COVID-19 se transmite por microgotas emitidas al respirar, hablar, toser o estornudar, o por contacto estrecho. Se ha intentado mitigar mediante diferentes medidas, que incluyen varias formas de distanciamiento social como reducciones de movilidad voluntarias, cuarentenas totales o parciales, teletrabajo, distancias mínimas entre personas, etc. Se han fomentado además distintas medidas de higiene como el uso de mascarillas, lavado frecuente de manos, ventilación de espacios, entre otras.

Con respecto a las medidas farmacéuticas, durante 2020 se estudiaron y desarrollaron varios intentos de vacunas, utilizando conocimientos adquiridos en la lucha contra virus como SARS-CoV y MERS-CoV. En 2021 se comenzó la vacunación masiva y hacia marzo de 2022, más de la mitad de la población mundial ha recibido al menos una dosis, con países como China y Singapur teniendo más de 85\% de su población vacunada según datos publicados en \textit{Our World in Data} \cite{Mathieu2021}.

Con respecto a características particulares de la enfermedad, se conoce de la existencia de casos sin síntomas o con síntomas leves y que también transmiten el virus, especialmente si no son reportados \cite{Li2020c}\cite{Byambasuren2020}\cite{Gao2021}. Su tiempo de incubación ha sido estimado en poco más de 5 días (5.1 días, Intervalo de Confianza del 95\% de 4.5 a 5.8 días según \cite{Lauer2020}; 5.2 días, IC 95\% de 4.1 a 7.0 días según \cite{Li2020d}). 

El impacto de la enfermedad ha sido heterogéneo entre la población. Se sabe que la edad es un factor a considerar; la susceptibilidad a la infección y la probabilidad de desarrollar síntomas luego del contagio aumentan con la edad \cite{Davies2020}. Similarmente, tanto el porcentaje de infectados cuya gravedad requiere hospitalización como el porcentaje que termina falleciendo es más alto entre los adultos mayores \cite{Verity2020}. El nivel socioeconómico tampoco puede ser ignorado; el menor acceso a la salud \cite{Wang2020}, sumado al desempleo y la existencia de enfermedades crónicas previas \cite{Ahmed2020} son un agravante que hace más vulnerables a los niveles socioeconómicos más pobres.

Los datos de movilidad han sido utilizados ampliamente para modelar el avance de la enfermedad \cite{Lai2020}\cite{Kraemer2020}\cite{Chinazzi2020}; en la mayoría de los países la movilidad explica una parte importante de la variaciones en la transmisibilidad \cite{Nouvellet2021}. \cite{Chang2021} nota como las diferencias en movilidad explican las diferencias en transmisión en diferentes grupos económicos, notando que además las clases de menor nivel socioeconómico se enfrentan a ambientes más riesgosos.

% El impacto de la enfermedad ha sido heterogéneo entre la población. Se han señalado al factor socioeconómico \cite{Ahmed2020} como un agravante... y son conocidos los efectos de la enfermedad en los adultos mayores.

% Los datos de movilidad han sido utilizados ampliamente para modelar el avance de la enfermedad \cite{Lai2020}\cite{Oliver2020}, se correlacionan bien con los contactos \cite{Prem2020} y la transmisibilidad \cite{Nasan2021} (ver bien Nasan). \cite{Chang2021} nota como las diferencias en movilidad explican las diferencias en transmisión en diferentes grupos económicos, notando que además las clases de menor nivel socioeconómico se enfrentan a ambientes más riesgosos.


\subsection{El brote de COVID-19 en Santiago de Chile }

La Región Metropolitana (RM) contiene a la ciudad de Santiago, la capital de Chile y su centro administrativo. Está formada por 52 comunas, las cuales varían significativamente en términos socioeconómicos. Su población es de unos \(7.3\) millones de personas, un 40\% aproximadamente de la población del país.

%[\url{https://www.gob.cl/coronavirus/gestionpandemia/}]
Las primeras medidas tomadas por el Gobierno de Chile para enfrentar la pandemia se centraron en fortalecer la capacidad hospitalaria: aumento en la cantidad de camas UCI, integración de los sistemas de salud público y privado, fortalecimiento de la capacidad de testeo y trazabilidad, preparación de residencias sanitarias, importación de material y equipamiento sanitario.

Se implementaron además medidas de distanciamiento social como la prohibición de eventos de gran concurrencia y el cierre de instituciones educativas. Se llevaron a cabo cuarentenas localizadas, hasta entrar en cuarentena total a mediados de mayo de 2020. En julio, una vez superado el primer \textit{peak} de casos, se comienza a aplicar el Plan Paso a Paso, una estrategia de desconfinamiento  gradual que sigue operando con modificaciones hasta la fecha.

Ya a fines de mayo de 2020, \cite{Olivares2020} había hecho notar "la dificultad de implementar cuarentenas obligatorias en comunas de menores recursos, donde la gente vive al día y genera sus ingresos con su trabajo diario", sugiriendo "que al aplicar las cuarentenas en comunas de menores niveles socioeconómicos, el gobierno las complemente con herramientas que apoyen a sus habitantes y provea de recursos para cubrir necesidades urgentes". Algunas medidas tomadas en esta dirección fueron la entrega de bonos e ingresos suplementarios, beneficios para el pago de servicios básicos, canastas de alimentos, etc.

Sin embargo, en \cite{Bennett2021} se muestra que, si bien las cuarentenas fueron efectivas en las comunas de mayores ingresos, no parecen tener un efecto significativo en las comunas de menores ingresos. Atribuye esto en parte a las diferencias en el cumplimiento de las medidas en términos de movilidad, así como a las diferentes capacidades de testeo. Estos resultados son similares a los obtenidos en \cite{Gozzi2021}. En \cite{Mena2021} se explora aún más este tópico, encontrando además una fuerte correlación entre el nivel socioeconómico y la mortalidad, que afecta aún más a los menores de 40 años.

El 3 de febrero de 2021 comienza el proceso de vacunación masiva de la población \cite{MINSAL2021}, el cual se realiza en su mayor parte con las vacunas Sinovac, Pfizer-BioNTech y Oxford-AstraZeneca. Hacia marzo de 2022 se está aplicando la cuarta dosis de refuerzo; Chile se encuentra entre los países con mayor porcentaje de personas vacunadas, rondando el 90\% de la población inoculada con dos dosis, similar a países como Canadá y Singapur \cite{Mathieu2021}.

El Ministerio de Salud ha liberado varios conjuntos de datos, que incluyen series de tiempo de casos confirmados, UCI, fallecidos, varias de ellas agregadas por edad o comuna. Están disponibles en el GitHub del Ministerio de Ciencia \cite{MINCIENCIA}. 



% Literatura relevante: se va a pasar todo a antecedentes
%\section{Revisión de literatura} \label{sec:literatura} 

%% Índice 
% El modelo que me interesa. 
% En qué contextos se ha usado. 
% Aquí hay un vacío, ya que la versión con ambientes no se ha ocupado nunca creo. 
% Justificación de por qué es interesante usarla. 

% Con respecto al covid y santiago (los antecedentes de mi caso de estudio) 
% Cómo se ha estudiado el Covid? Hay muchísimos modelos disponibles, dar algunos ejemplos. 
% Qué se ha hecho con respecto al covid en Chile? Qué modelos se han usado? Qué cosas se han descubierto? Creo que el artículo de Mena debería ir aquí. También debería volver a mirar los seminarios, ahí hay harta info. Buscar los papers.
% Lo mismo pero en Santiago. 
% Aquí sería interesante encontrar un nicho también... algo que no se haya intentado, algo que los modelos no han incorporado... etc 

% Con respecto a las metodologías específicas que usaré.
% Qué otras metodologías existen? De qué otra manera podría estimar los parámetros de esto.? 
% Para filtro de kalman 
% Cómo se ha usado en otras epidemias 
% Cómo se ha usado en el caso del Covid
% Algún vacío en la literatura? 

% Revisar el artículo que estaba leyendo (Approximation rates), ellos tienen una revisión de literatura que me parece completa y bien escrita. 

Varios artículos posteriores han implementado el primer modelo basado en tiempos de residencia propuesto por \cite{Bichara2015} [Revisar si es verdad]. Sin embargo, nadie ha usado aún el segundo modelo propuesto, donde se utilizan ambientes en lugar de clases. Este modelo es interesante pues... 


%%%%%%%%%%%%%%%%%%%%%%%%%%%%%%%%%%%%%%%%%%%%%%%%
% Covid en Santiago %%%%%%%%%%%%%%%%%%%%%%%%%%%%
%%%%%%%%%%%%%%%%%%%%%%%%%%%%%%%%%%%%%%%%%%%%%%%%


Mucha investigación se ha llevado a cabo con respecto al Covid19 estos últimos meses. Se han estudiado modelos como ... y ... 

En Santiago, en particular ... ... 


%%%%%%%%%%%%%%%%%%%%%%%%%%%%%%%%%%%%%%%%%%%%%%%%
% KALMAN %%%%%%%%%%%%%%%%%%%%%%%%%%%%%%%%%%%%%%%
%%%%%%%%%%%%%%%%%%%%%%%%%%%%%%%%%%%%%%%%%%%%%%%%

Para la estimación de parámetros se ha decidido usar el filtro de Kalman. Esto pues... % (buscar justificación dentro de los mismo artículos que he revisado)

Las distintas variantes del Filtro de Kalman se han usado previamente en epidemiología para estimación de parámetros, tanto fijos como cambiantes en el tiempo, trabajando con datos relacionados a SIDA \cite{Cazelles1997}, ébola \cite{Ndanguza2017}, dengue \cite{Torres-Signes2021}, tuberculosis \cite{Narula2016}, etc. % Creo que puedo dar más detalles de cada uno. 

En este último tiempo ha sido incorporado al análisis de datos de COVID-19. Se ve el uso repetido del EnKF en predicción [Hameni et al], \cite{Yang2020}, \cite{Song2021}. En \cite{Arroyo-Marioli2021} y \cite{Hasan2020} [Hasan et al (buscar año)] usan filtro de kalman y EKF respectivamente para estimar el número de reproducción efectivo \( \mathcal{R}_t \). En \cite{Rajaei2021} se usa EKF para desarrollar un control robusto basado en distanciamiento social, hospitalización y tratamiento, y vacunación. En \cite{Bansal2021} se usa \textit{Fractional Order Calculus}, un técnica de filtraje de la que EKF puede verse como un caso particular, para estimar transmisibilidad.


% 11/05/2020 - [Gomez-Exposito et al] EKF para estimar un indicador predicción 
% 12/06/2020 - [Aslam] Lo usa para hacer forecasting con datos de Pakistan, sin especificar más allá los detalles...
% 11/07/2020 - [Hameni et al] Usan EnKF para predicciones... usando un modelo basado en SIR.
% 05/08/2020 - [Yang et al] Usan EnKF para predicción de corto plazo.
% 17/11/2020 - [Arroyo-Marioli] Usan el filtro de kalman para estimar el Rt
% 2 December 2020 - [Li, Zhao, ..] Usan EnKF para forecasting.
% 13/01/2021 - [Hasan et al] Usan EKF para estimar el número de reproducción efectivo \( \mathcal{R}_t \).
% 15/02/2021 - [Bansal] Usa un técnica que engloba KF
% 22/03/2021 - [Song et al] EKF 
% 30/03/2021 - [Rajaei et al] Control robusto


% Elección de parámetros 

Mi impresión es que la elección de parámetros es verdaderamente un vacío en la literatura. Cuando leo los artículos no me parece que den una buena justificación de por qué lo hacen así. Me interesa leer a los tipos que usaron el filtro Hinfty, porque creo que debido a las incertezas que hay en todo el proceso de modelar esta cuestión, tiene mucho sentido usarlo.

