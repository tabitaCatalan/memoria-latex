\chapter{Discusión} \label{chap:discus}

A continuación se ofrece una discusión con respecto al trabajo realizado. La sección \ref{dis:caso} comenta los resultados obtenidos para el caso de estudio y para las situaciones hipotéticas, resaltando las limitaciones derivadas de los supuestos y simplificaciones hechos a lo largo del modelamiento, y presentando posible interpretaciones. La sección \ref{dis:metod} comenta la metodología seguida, presenta algunos enfoques alternativos con que se podría haber abordado el trabajo, expone posibles extensiones y discute el cumplimiento de los objetivos.

\section{Acerca del caso de estudio}\label{dis:caso}

En primer lugar, es necesario recordar los supuestos y simplificaciones hechos a la hora de modelar el caso de estudio. Por un lado, se consideró que la población en la Región Metropolitana se mantenía constante, despreciando efectos como la natalidad, mortalidad debido a causas naturales y efectos migratorios entre ciudades. Por otro lado, algunos factores relevantes a la transmisión del COVID-19 fueron dejados fuera al utilizar un modelo SEIR, como la existencia de casos sin síntomas o con síntomas leves, que pueden o no ser reportados y que también transmiten el virus \cite{Li2020c}\cite{Byambasuren2020}\cite{Gao2021}. El supuesto de que todas las clases socioeconómicas tienen la misma tasa de incubación y recuperación es un factor que debe ser considerado, especialmente al tener en cuenta el estudio de sensibilidad presentado en \ref{subsec:sensigamma}.

La movilidad... los casos no reportados

El uso del filtro de Kalman tiene, desde luego, limitaciones; teóricamente solo puede justificarse en el caso lineal, y el uso del filtro extendido lleva a errores numéricos que podrían mitigarse usando un filtro más apto para tratar con no linealidades, como el filtro \textit{unscented}. El estudio del caso sintético muestra también que el filtro tiene dificultades para estimar los valores sanitarios al comienzo de la pandemia, cuando hay muy pocos casos, por lo que los resultados en la fase inicial son poco confiables. Los estudios de sensibilidad mostraron además las diferencias en las estimaciones, dependiendo de los valores de varianzas elegidas como parámetros del filtro.

El número de casos detectados no es la mejor observación; se ha hablado de cómo la capacidad de detectar los nuevos contagios difiere entre clases socioeconómicas \cite{Mena2021}, sugiriendo que el número de fallecidos es una mejor medida. Esto sin embargo, sugiere que el número de casos está subestimado para las clases más vulnerables, lo que solo contribuye a resaltar los resultados obtenidos.

Una medida de gran relevancia que no fue considerada en el modelo fue la vacunación. Esta, sin embargo, debería verse directamente reflejada en una disminución importante del factor sanitario una vez que se vacuna gran parte de la población.

%%% Factor sanitario 
Con respecto a los resultados obtenidos, se destaca el hecho de que el factor sanitario de la clase más acomodada se mantiene por debajo de las demás clases. Ese resultado es constante a través de los análisis de sensibilidad. Se intentó inicialmente obtener un factor sanitario común a todas las clases, lo que sugeriría que todos enfrentan el mismo riesgo en el exterior, pero no fue posible obtener resultados razonables.

Todo esto sugiere que las diferencias en movilidad por sí solas no son suficientes para explicar las diferencias en la cantidad de contagios. Las clases socioeconómicas más vulnerables no solo no pueden guardar las cuarentenas de la misma forma que las clases acomodadas, sino que además, al salir al exterior, se encuentran en ambientes más riesgosos, ya sea por estar en ambientes más concurridos, o por no utilizar adecuadamente las medidas de seguridad como el lavado de manos y el uso correcto de mascarillas. Estos resultados son consistentes con los encontrados por \cite{Chang2021}. Otras posibles explicaciones podrían ser las diferencias en la capacidad de testeo y trazabilidad \cite{Bennett2021}\cite{Mena2021}.

Con respecto al ajuste del factor sanitario, a pesar de la gran incerteza que surge de utilizar valores grandes de covarianza, el caso sintético sugiere que se obtienen mejores resultados de esta forma. Los resultados deben considerar además los estudios de sensibilidad de la subsección \ref{subsec:sensigamma}. Mientras la hipótesis \(\gamma_{Ei} = \gamma_E\) y \(\gamma_{Ii} = \gamma_I\) sea válida, es decir, que las tasas de incubación y remoción son iguales en cada clase, se puede esperar que las estimaciones del factor sanitario \(\alpha_i\) se expandan o contraigan uniformemente. La tasa de exposición \(\gamma_E\) no tiene gran relevancia, sin embargo, la de recuperación \(\gamma_I\) puede hacer variar considerablemente los resultados. Ahora bien, para subestimar el factor sanitario, se debería tener un tiempo de recuperación mayor, pero esto es improbable considerando que el acceso a salud es superior en las clases más acomodadas.

Destaca el factor sanitario en el periodo mayo-junio de 2021, donde se observa que la clase más acomodada consigue una reducción mucho más marcada que las demás. Una posible hipótesis sería atribuir esto al avance del proceso de vacunación en Chile, que para esas fechas ya había superado el 50\% de población vacunada en la Región Metropolitana (ver figura \ref{img:cmm-vacunados}); ¿fue el proceso de vacunación más marcado en las comunas acomodadas? La distribución de vacunas ocurrió por tramos de edad, de mayores a jóvenes, y la distribución de adultos mayores en las comunas acomodadas es mayor \cite{LaTercera}, pero es algo que habría que estudiar mejor.

%% Casos hipotéticos 
Con respecto a los casos hipotéticos, se observan dos escenarios que dependen del valor del riesgo \(\beta_{\text{exterior}}\). Para este análisis ese necesario considerar el aporte a la derivada de cada uno de los ambientes, calculado en \ref{}. 
Por un lado, al usar un valor notoriamente mayor que \(1\) (\(\beta_{\text{exterior}} = 10\), el aporte a la derivada dado por el ambiente \(\text{hogar}\) es despreciable en comparación al dado por el \(\text{exterior}\). Esto quiere decir que prácticamente todos los contagios ocurren cuando la gente sale de su hogar. Esto, sin embargo, no es consistente con \cite{Ferguson2020}, quien afirma que aproximadamente 1/3 de los contagios ocurren en el hogar.

En esta situación se observa que las medidas de cuidado son sumamente relevantes; los casos ``Cuidado extra sin cuarentena'' y ``Cuidado normal con cuarentena fuerte'' son muy similares, salvo que las mejores medidas de cuidado logran prevenir un segundo \textit{peak} en agosto de 2021. En los otros casos de ``Cuidado extra'', con cuarentena normal o fuerte, se logra controlar la pandemia dentro de los primeros seis meses.

Por otro lado, cuando se usa un valor ligeramente mayor a \(1\) (\(\beta_{\text{exterior}} = 1.2\) por ejemplo) el aporte del ambiente \texttt{hogar} a la derivada ya es perceptible \texttt{exterior} (para este caso en particular, es unas 10 veces menor). El estudio de los casos hipotéticos, sin embargo, revela que la cuarentena casi no influya en nada en los resultados finales, y que todo depende de las medidas de cuidado.

El modelo en ambos escenarios coincide en que si fuera posible que todos se cuidaran al mismo nivel que el grupo más acomodado, se podría controlar el virus incluso sin la exigencia de cuarentenas.

\section{Acerca de la metodología}\label{dis:metod}

  

% Cómo influyeron las decisiones tomadas en los resultados? 

% Matriz de tiempos de residencia - EOD2012

Una de las diferencias distintivas de este enfoque es utilizar los datos de movilidad para estimar tiempos de residencia, siguiendo la forma lagrangiana. Esta no era la única forma posible, los datos de la Encuesta Origen Destino contienen información detallada de viajes, incluyendo comunas de salida y llegada. Usando esto, una forma alternativa de usar la Encuesta habría sido usar sus datos como movilidad basa entre las distintas comunas, y aprovechar los datos de movilidad en pandemia, como los del ISCI presentados en la sección \ref{}, para modificar estos viajes, siguiendo una metodología similar a \cite{Lai2020}.


De seguir esta forma de trabajo, habría sido posible definir algo análogo al factor sanitario, para cada comuna. El no tener una distinción entre clases y ambientes sin embargo, impone el tener que trabajar únicamente con divisiones espaciales, como zonas censales, comunas y agrupaciones de comunas, dificultando el trabajo con grupos etarios.

Ahora, si bien la metodología propuesta sí permite sortear esta dificultad, no se elimina completamente, puesto que sigue siendo necesario estimar los tiempos de residencia para cada una de las clases, y la forma en que estos variaron en cada ambiente no fue uniforme en distintos grupos etarios. A modo de ejemplo, podría ser posible considerar grupos etarios, puesto que existen series de tiempo de número de contagios agrupados por edad. La variación de movilidad por grupo etario, sin embargo, no es tan fácil de obtener, considerando que hay grupos mal representados en los datos disponibles; los datos de ISCI por ejemplo, se derivan del uso de torres de telefonía móvil, que en el caso de niños y adultos mayores podría no ser adecuado.

Al intentar agregar más ambientes al análisis, surgen problemas similares. Los datos más cercanos a movilidad por ambiente en pandemia son los provistos por Google, presentados en la sección \ref{}. Estos datos están agrupados a nivel regional/provincial, pero se sabe \cite{Olivares2020} que no son uniformes entre distintos niveles socioeconómicos; el teletrabajo es mucho más común en niveles altos, mientras que un porcentaje importante de la población más vulnerable tiene trabajos esenciales que no pueden detenerse en pandemia \cite{}.

Debido a este problema, la Encuesta Origen Destino terminó jugando un rol muy secundario en el trabajo, ya que su utilidad se vio relegada a una simple estimación del tiempo que pasaban en el hogar, antes de la pandemia, los distintos grupos socioeconómicos, y habría sido de mejor aprovechada siguiendo la metodología euleriana antes propuesta. 

La metodología propuesta, sin embargo, presenta la ventaja de ser utilizable con menos datos de movilidad disponible, por lo que a aplicable incluso en regiones donde no se tenga el detalle de las tasas de movilidad entre cada par de zonas; solo es necesario saber cuánto se mueve la gente, no hacia dónde lo hace, debido a que todos interactúan en un ambiente \texttt{exterior} común.

Una desventaja no prevista en un principio es la estimación de los riesgos específicos por ambiente; no es tan clara como se supuso. Esto quedó en evidencia al observar los resultados de la sección \ref{sec:evalmodel-casoshipot}. Estimaciones de la forma: ``aproximadamente 1/3 de los contagios ocurren en el hogar'' \cite{Ferguson2020} no se traducen directamente en asignar \( 2 \beta_{\text{hogar}} = \beta_{\text{exterior}}\).

% Conclusión del análisis de sensibilidad

