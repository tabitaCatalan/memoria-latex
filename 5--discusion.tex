\chapter{Discusión} \label{chap:discus}

Meta: los objetivos específicos apuntan a conseguir el objetivo general? Se cumplió el objetivo? Qué puede decirse del framework.

\section{Acerca del caso de estudio}

Con respecto a las limitaciones, es necesario recordar los supuestos y simplificaciones. Por un lado, se consideró que la población en la región metropolitana se mantenía constante, despreciando efectos como la natalidad, mortalidad debido a causas naturales y efectos migratorios entre ciudades. Por otro lado, algunos factores relevantes a la transmisión del Covid19 fueron dejados fuera al utilizar un modelo SEIR, como la importancia de los casos no detectados \cite{Li2020c} o la posibilidad de reinfección \cite{}. El supuesto de que todas las clases socioeconómicas tienen la misma tasa de incubación y recuperación es un factor a considerar, y que sería ciertamente excesivo al haber utilizado grupos etarios \cite{}. Este supuesto toma relevacia al considerar los estudios de sensibilidad; al considerar grupos con tasas muy distintas, nada asegura que se mantenga la relación entre los factores sanitarios.

El uso del filtro de Kalman tiene desde luego limitaciones. Teóricamente solo puede justificarse en el caso lineal, y el uso del filtro extendido lleva a errores numéricos que podrían mitigarse usando un filtro más apto para tratar con no linealidades, como el filtro \textit{unscented}. El estudio del caso sintético muestra también que el filtro tiene dificultades para estimar los valores sanitarios al comienzo de la pandemia, cuando hay muy pocos casos, por lo que los resultados en la fase inicial son poco confiables. Desde luego está la estimación de las tasas, que genera variaciones importantes en los estados, y de las covarianzas... 

Los datos utilizados como observaciones no son los ideales; se ha mencionado de la capacidad de detectar los casos contagiados difiere entre clases socioeconómicas \cite{Mena2021}, sin embargo, eso sugiere que el número de casos acumulados está subestimado para las clases más vulnerables, y que de usar los reales se conseguirían diferencias aún más grandes en el factor sanitario.

Una medida de gran relevancia que no fue considerada en el trabajo fue la vacunación. Esta, sin embargo, debería verse directamente reflejada como una disminución importante del factor sanitario una vez que se vacuna gran parte de la población.

%%% Factor sanitario 
Con respecto al factor sanitario, el hecho de que la clase más acomodada mantiene un factor sanitario por debajo de las demás clases es algo que se mantiene constante a través de los estudios de sensibilidad. Se intentó obtener un factor sanitario común a todas las clases, lo que sugeriría que todos enfrentan el mismo riesgo en el exterior, pero no fue posible obtener resultados razonables, .

Estos resultados sugieren que las diferencias en movilidad, las cuales han sido previamente mencionadas como una explicación de las diferencias en la incidencia entre los distintos grupos socioeconómicos \cite{Mena2021}\cite{Gozzi2021} [Revisar], por sí solas no son suficientes para explicar las diferencias en la cantidad de contagios.

Las clases socioeconómicas más vulnerables no solo no pueden guardar la cuarentenas de la misma forma que las clases acomodadas, sino que además, al salir al exterior, se encuentran en ambientes más riesgosos, ya sea por estar en ambientes más concurridos, o por no utilizar adecuadamente las medidas de seguridad como el lavado de manos y el uso correcto de mascarillas. Estos resultados son consistentes con los encontrados por \cite{Chang2021}.

Destaca la el factor sanitario en el periodo mayo-junio de 2021, donde se observa que la clase más acomodada consigue una reducción mucho más marcada que las demás. Una posible hipótesis sería atribuir esto al avance del proceso de vacunación en Chile (ver figura \ref{img:cmm-vacunados}); ¿fue el proceso de vacunación más marcado en las comunas acomodadas? La distribución de vacunas ocurrió por tramos de edad, de mayores a jóvenes, y la distribución de adultos mayores en las comunas acomodadas es mayor.

Otra posible explicación nace del análisis de sensibilidad realizado en 

%% Casos hipotéticos 
Con respecto a los casos hipotéticos, se observan dos escenarios que dependen del valor del riesgo \(\beta_{\text{exterior}}\). Para este análisis ese necesario considerar el aporte a la derivada de cada uno de los ambientes, calculado en \ref{}. 
Por un lado, al usar un valor notoriamente mayor que \(1\) (\(\beta_{\text{exterior}} = 10\), el aporte a la derivada dado por el ambiente \(\text{hogar}\) es despreciable en comparación al dado por el \(\text{exterior}\). Esto quiere decir que prácticamente todos los contagios ocurren cuando la gente sale de su hogar. Esto, sin embargo, no es consistente con \cite{Ferguson2020}, quien afirma que aproximadamente 1/3 de los contagios ocurren en el hogar. Incluso en esta situación, se observa que las medidas de cuidado son sumamente relevantes, pudiendo prevenir un segundo \textit{peak} como muestra el caso 3.

Por otro lado, cuando se usa un valor ligeramente mayor a \(1\) (\(\beta_{\text{exterior}} = 1.2\) por ejemplo) el aporte del ambiente \texttt{hogar} a la derivada ya es perceptible \texttt{exterior} (para este caso en particular, es unas 10 veces menor). El estudio de los casos hipotéticos, sin embargo, revela que la cuarentena casi no influya en nada en los resultados finales, y que todo depende de las medidas de cuidado.

El modelo en ambos escenarios coincide, sin embargo, en que si fuera posible que todos se cuidaran al mismo nivel que el grupo más acomodado, se podría controlar el virus incluso sin la exigencia de cuarentenas.

\section{Acerca de la metodología}

  

Cómo influyeron las decisiones tomadas en los resultados? 
- Lagrangiano vs euleriano 
- Usar la matriz de tiempos de residencia... aquí no, pero en otro contexto podría serlo.... Se podría haber usado de otra forma, con el enfoque de \cite{Chang2021}, la EOD tiene los viajes bases y la movilidad daba cómo modificarlos.

No se aprovechó completamente la matriz de tiempos de residencia. Si bien se pudo calcular los tiempos en una gran variedad de ambientes, la forma de actualizar esa matriz en el tiempo presenta un desafío. Algunas dificultades incluyen la baja representatividad de los datos de algunas clases socieoconómicas, como los niños en los datos de movilidad obtenidos con las redes de telefonía. 

La existencia de la EOD2012 ofrecía en primer lugar la posibilidad de un enfoque euleriano, ya que los datos de viajes entre comunas estableciendo una movilidad base, agrupado con los datos de movilidad modificada en pandemia ofrecían la posibilidad de usar un modelo muy similar al empleado por \cite{Lai2020}.


Las ventajas de este enfoque lagrangiano se destacarían más al modelar una ciudad con menor disponibilidad de datos....


% Conclusión del análisis de sensibilidad. 
Mientras la hipótesis \(\gamma_{Ei} = \gamma_E\) y \(\gamma_{Ii} = \gamma_I\) sea válida, es decir, que las tasas de incubación y remoción son iguales en cada clase, se puede esperar que las estimaciones del factor sanitario \(\alpha_i\) conserven la proporción entre ellas, de forma que, aunque no sean exactas, sí sea posible hacer afirmaciones del tipo ``la clase \(1\) enfrentó un 20\% más de riesgo que la clase \(2\) en este período de tiempo''. 

Si esto falla, entonces no se puede esperar que los factores sanitarios se expandan o contraigan uniformemente. El análisis hecho en la subsección \ref{subsec:sensigamma} muestra que el valor \(\gamma_I\) es el que más influye en esta contracción.

Con respecto al ajuste del factor sanitario, a pesar de la gran incerteza que surge de utilizar valores grandes de covarianza, el caso sintético sugiere que se obtienen mejores resultados de esta forma.

- Comparación con la literatura de la metodología 
Se cumplieron los objetivos??



El trabajo realizado ofrece varias posibles extensiones. Por un lado, la metodología empleada es aplicable con mínimas modificaciones a una gran variedad de modelos, con otros compartimientos o variables observadas. También era posible, puesto que los datos están disponibles para el caso de la región Metropolitana, aplicar el modelo por comuna en lugar de por grupos socioeconómicos, aunque esto requería más tiempo de cómputo. 

Probar el modelo con otros ambientes es una posibilidad interesante. Algunas sugerencias son incluir ambientes de trabajo, compras, etc, utilizando datos como los ofrecidos por Google (ver sección \ref{}).

La librería ModelingToolkit.jl \cite{}, aun en una fase inicial de desarrollo, ofrece muchas posibilidades a la modelación y vale la pena verlas en más detalle.