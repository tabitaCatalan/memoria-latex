\chapter{Discusión} \label{chap:discus}

A continuación se comentan los resultados obtenidos utilizando la metodología propuesta para el caso de estudio, y se discuten interpretaciones y posibles extensiones.

\section{Acerca del caso de estudio}

Con respecto a las limitaciones, es necesario recordar los supuestos y simplificaciones. Por un lado, se consideró que la población en la región metropolitana se mantenía constante, despreciando efectos como la natalidad, mortalidad debido a causas naturales y efectos migratorios entre ciudades. Por otro lado, algunos factores relevantes a la transmisión del Covid19 fueron dejados fuera al utilizar un modelo SEIR, como la importancia de los casos no detectados \cite{Li2020c} o la posibilidad de reinfección \cite{}. El supuesto de que todas las clases socioeconómicas tienen la misma tasa de incubación y recuperación es un factor a considerar, y que sería ciertamente excesivo al haber utilizado grupos etarios \cite{}. Este supuesto toma relevacia al considerar los estudios de sensibilidad; al considerar grupos con tasas muy distintas, nada asegura que se mantenga la relación entre los factores sanitarios.

El uso del filtro de Kalman tiene desde luego limitaciones. Teóricamente solo puede justificarse en el caso lineal, y el uso del filtro extendido lleva a errores numéricos que podrían mitigarse usando un filtro más apto para tratar con no linealidades, como el filtro \textit{unscented}. El estudio del caso sintético muestra también que el filtro tiene dificultades para estimar los valores sanitarios al comienzo de la pandemia, cuando hay muy pocos casos, por lo que los resultados en la fase inicial son poco confiables. Desde luego está la estimación de las tasas, que genera variaciones importantes en los estados, y de las covarianzas... 

Los datos utilizados como observaciones no son los ideales; se ha mencionado de la capacidad de detectar los casos contagiados difiere entre clases socioeconómicas \cite{Mena2021}, sin embargo, eso sugiere que el número de casos acumulados está subestimado para las clases más vulnerables, y que de usar los reales se conseguirían diferencias aún más grandes en el factor sanitario.

Una medida de gran relevancia que no fue considerada en el trabajo fue la vacunación. Esta, sin embargo, debería verse directamente reflejada como una disminución importante del factor sanitario una vez que se vacuna gran parte de la población.

%%% Factor sanitario 
Con respecto al factor sanitario, el hecho de que la clase más acomodada mantiene un factor sanitario por debajo de las demás clases es algo que se mantiene constante a través de los estudios de sensibilidad. Se intentó obtener un factor sanitario común a todas las clases, lo que sugeriría que todos enfrentan el mismo riesgo en el exterior, pero no fue posible obtener resultados razonables, .

Estos resultados sugieren que las diferencias en movilidad, las cuales han sido previamente mencionadas como una explicación de las diferencias en la incidencia entre los distintos grupos socioeconómicos \cite{Mena2021}\cite{Gozzi2021} [Revisar], por sí solas no son suficientes para explicar las diferencias en la cantidad de contagios.

Las clases socioeconómicas más vulnerables no solo no pueden guardar la cuarentenas de la misma forma que las clases acomodadas, sino que además, al salir al exterior, se encuentran en ambientes más riesgosos, ya sea por estar en ambientes más concurridos, o por no utilizar adecuadamente las medidas de seguridad como el lavado de manos y el uso correcto de mascarillas. Estos resultados son consistentes con los encontrados por \cite{Chang2021}.

Destaca la el factor sanitario en el periodo mayo-junio de 2021, donde se observa que la clase más acomodada consigue una reducción mucho más marcada que las demás. Una posible hipótesis sería atribuir esto al avance del proceso de vacunación en Chile (ver figura \ref{img:cmm-vacunados}); ¿fue el proceso de vacunación más marcado en las comunas acomodadas? La distribución de vacunas ocurrió por tramos de edad, de mayores a jóvenes, y la distribución de adultos mayores en las comunas acomodadas es mayor.

Mientras la hipótesis \(\gamma_{Ei} = \gamma_E\) y \(\gamma_{Ii} = \gamma_I\) sea válida, es decir, que las tasas de incubación y remoción son iguales en cada clase, se puede esperar que las estimaciones del factor sanitario \(\alpha_i\) conserven la proporción entre ellas, de forma que, aunque no sean exactas, sí sea posible hacer afirmaciones del tipo ``la clase \(1\) enfrentó un 20\% más de riesgo que la clase \(2\) en este período de tiempo''. 

Si esto falla, entonces no se puede esperar que los factores sanitarios se expandan o contraigan uniformemente. El análisis hecho en la subsección \ref{subsec:sensigamma} muestra que el valor \(\gamma_I\),la tasa de remoción del sistema, la tasa de personas que dejan de contagiar, es el que más influye en esta contracción;. Durante todo el análisis se supuso que este valor era igual para todas las clases (\(\gamma_{Ii} = \gamma_I\) para cada \(i\)), lo que sugiere una segunda consideración a los resultados; es posible que la tasa de remoción sea distinta para cada nivel socioeconómico, ya que incluye a las personas que fallecen debido a la enfermedad o que se recuperan, y ambas cosas dependen de factores clínicos, los que podrían variar en distintos niveles socioeconómicos \cite{Mena2021}. El hecho de que se vea una reducción marcada en el factor para la clase acomodada, específicamente en los meses posteriores a la vacunación, hace que la primera siga siendo una opción plausible.

Con respecto al ajuste del factor sanitario, a pesar de la gran incerteza que surge de utilizar valores grandes de covarianza, el caso sintético sugiere que se obtienen mejores resultados de esta forma.

%% Casos hipotéticos 
Con respecto a los casos hipotéticos, se observan dos escenarios que dependen del valor del riesgo \(\beta_{\text{exterior}}\). Para este análisis ese necesario considerar el aporte a la derivada de cada uno de los ambientes, calculado en \ref{}. 
Por un lado, al usar un valor notoriamente mayor que \(1\) (\(\beta_{\text{exterior}} = 10\), el aporte a la derivada dado por el ambiente \(\text{hogar}\) es despreciable en comparación al dado por el \(\text{exterior}\). Esto quiere decir que prácticamente todos los contagios ocurren cuando la gente sale de su hogar. Esto, sin embargo, no es consistente con \cite{Ferguson2020}, quien afirma que aproximadamente 1/3 de los contagios ocurren en el hogar. Incluso en esta situación, se observa que las medidas de cuidado son sumamente relevantes, pudiendo prevenir un segundo \textit{peak} como muestra el caso 3.

Por otro lado, cuando se usa un valor ligeramente mayor a \(1\) (\(\beta_{\text{exterior}} = 1.2\) por ejemplo) el aporte del ambiente \texttt{hogar} a la derivada ya es perceptible \texttt{exterior} (para este caso en particular, es unas 10 veces menor). El estudio de los casos hipotéticos, sin embargo, revela que la cuarentena casi no influya en nada en los resultados finales, y que todo depende de las medidas de cuidado.

El modelo en ambos escenarios coincide, sin embargo, en que si fuera posible que todos se cuidaran al mismo nivel que el grupo más acomodado, se podría controlar el virus incluso sin la exigencia de cuarentenas.

\section{Acerca de la metodología}

  

% Cómo influyeron las decisiones tomadas en los resultados? 

% Matriz de tiempos de residencia - EOD2012

Una de las diferencias distintivas de este enfoque es utilizar los datos de movilidad para estimar tiempos de residencia, siguiendo la forma lagrangiana. Esta no era la única forma posible, los datos de la Encuesta Origen Destino contienen información detallada de viajes, incluyendo comunas de salida y llegada. Usando esto, una forma alternativa de usar la Encuesta habría sido usar sus datos como movilidad basa entre las distintas comunas, y aprovechar los datos de movilidad en pandemia, como los del ISCI presentados en la sección \ref{}, para modificar estos viajes, siguiendo una metodología similar a \cite{Lai2020}.


De seguir esta forma de trabajo, habría sido posible definir algo análogo al factor sanitario, para cada comuna. El no tener una distinción entre clases y ambientes sin embargo, impone el tener que trabajar únicamente con divisiones espaciales, como zonas censales, comunas y agrupaciones de comunas, dificultando el trabajo con grupos etarios.

Ahora, si bien la metodología propuesta sí permite sortear esta dificultad, no se elimina completamente, puesto que sigue siendo necesario estimar los tiempos de residencia para cada una de las clases, y la forma en que estos variaron en cada ambiente no fue uniforme en distintos grupos etarios. A modo de ejemplo, podría ser posible considerar grupos etarios, puesto que existen series de tiempo de número de contagios agrupados por edad. La variación de movilidad por grupo etario, sin embargo, no es tan fácil de obtener, considerando que hay grupos mal representados en los datos disponibles; los datos de ISCI por ejemplo, se derivan del uso de torres de telefonía móvil, que en el caso de niños y adultos mayores podría no ser adecuado.

Al intentar agregar más ambientes al análisis, surgen problemas similares. Los datos más cercanos a movilidad por ambiente en pandemia son los provistos por Google, presentados en la sección \ref{}. Estos datos están agrupados a nivel regional/provincial, pero se sabe \cite{Olivares2020} que no son uniformes entre distintos niveles socioeconómicos; el teletrabajo es mucho más común en niveles altos, mientras que un porcentaje importante de la población más vulnerable tiene trabajos esenciales que no pueden detenerse en pandemia \cite{}.

Debido a este problema, la Encuesta Origen Destino terminó jugando un rol muy secundario en el trabajo, ya que su utilidad se vio relegada a una simple estimación del tiempo que pasaban en el hogar, antes de la pandemia, los distintos grupos socioeconómicos, y habría sido de mejor aprovechada siguiendo la metodología euleriana antes propuesta. 

La metodología propuesta, sin embargo, presenta la ventaja de ser utilizable con menos datos de movilidad disponible, por lo que a aplicable incluso en regiones donde no se tenga el detalle de las tasas de movilidad entre cada par de zonas; solo es necesario saber cuánto se mueve la gente, no hacia dónde lo hace, debido a que todos interactúan en un ambiente \texttt{exterior} común.

Una desventaja no prevista en un principio es la estimación de los riesgos específicos por ambiente; no es tan clara como se supuso. Esto quedó en evidencia al observar los resultados de la sección \ref{sec:evalmodel-casoshipot}. Estimaciones de la forma: ``aproximadamente 1/3 de los contagios ocurren en el hogar'' \cite{Ferguson2020} no se traducen directamente en asignar \( 2 \beta_{\text{hogar}} = \beta_{\text{exterior}}\).

% Conclusión del análisis de sensibilidad

El framework propuesto puede aplicarse a una amplia variedad de modelos, compartimientos, parámetros y observaciones. Algunas posibles extensiones son considerar los efectos de la vacunación en los compartimientos, usar comunas en lugar de grupos socioeconómicos, usar un filtro que se comporte mejor frente a no linealidades (como el filtro \textit{unscented} o ``sin olor''), ampliar el modelo para considerar fallecidos, etc.

Un aporte lateral del trabajo realizado es el código realizado, que se encuentra disponible en línea en la plataforma GitHub. El repositorio \url{} contiene una implementación en Matlab del algoritmo de extracción de tiempos de residencia mencionado brevemente en \cite{Munizaga2011}. El repositorio \url{} tiene la librería KalmanFilter.jl, documentada, modular y extensible, desarrollada en el lenguaje Julia, que permite trabajar con Filtro de Kalman lineal, extendido y ``sin olor`` en sus versiones continuas y discretas, utilizando observadores lineales y no lineales. Finalmente, el repositorio \url{} contiene el código utilizado en el análisis del caso de estudio del Covid en Santiago. Este repositorio hace uso de la novedosa librería ModelingToolkit.jl \cite{}, aun en una fase inicial de desarrollo, la cual, de forma similar a Modellica o ... [insertar software] ofrece herramientas para el trabajo con modelos de ecuaciones diferenciales, cálculo simbólico de derivadas, de forma eficiente, y extensible.

% Comparación con la literatura de la metodología 
% Se cumplieron los objetivos?? Era atingentes los objetivos? 

Los objetivos específicos fueron logrados en su mayor parte, puesto que fue posible plantear una forma de estimar el factor sanitario y realizar una implementación para el caso de estudio. Sin embargo, era deseable poder incluir en el modelo ambientes como \texttt{trabajo}, \texttt{estudios}, \texttt{compras} o \texttt{transporte público}, o considerar grupos etarios, y esto no fue logrado.