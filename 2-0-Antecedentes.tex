\section{Antecedentes: El caso del COVID-19} \label{sec:antecedentes}

El COVID-19 se transmite por microgotas emitidas al respirar, hablar, toser o estornudar, o por contacto estrecho \cite{Dong2020}\cite{Greenhalgh2021}. Se ha intentado mitigar mediante diferentes medidas \cite{Flaxman2020}\cite{Castillo-Laborde2021}, que incluyen varias formas de distanciamiento social como reducciones de movilidad voluntarias, cuarentenas totales o parciales, teletrabajo, distancias mínimas entre personas, etc. Se han fomentado además distintas medidas de higiene como el uso de mascarillas, lavado frecuente de manos, ventilación de espacios, entre otras.

Con respecto a las medidas farmacéuticas, durante 2020 se estudiaron y desarrollaron varios intentos de vacunas, utilizando conocimientos adquiridos en la lucha contra virus como SARS-CoV y MERS-CoV. En 2021 se comenzó la vacunación masiva y hacia marzo de 2022, más de la mitad de la población mundial ha recibido al menos una dosis, con países como China y Singapur teniendo más de 85\% de su población vacunada según datos publicados en \textit{Our World in Data} \cite{Mathieu2021}.

Con respecto a características particulares de la enfermedad, se conoce de la existencia de casos sin síntomas o con síntomas leves y que también transmiten el virus, especialmente si no son reportados \cite{Li2020c}\cite{Byambasuren2020}\cite{Gao2021}. Su tiempo de incubación ha sido estimado en poco más de 5 días (5.1 días, Intervalo de Confianza del 95\% de 4.5 a 5.8 días según \cite{Lauer2020}; 5.2 días, IC 95\% de 4.1 a 7.0 días según \cite{Li2020d}). 

El impacto de la enfermedad ha sido heterogéneo entre la población. Se sabe que la edad es un factor a considerar; la susceptibilidad a la infección y la probabilidad de desarrollar síntomas luego del contagio aumentan con la edad \cite{Davies2020}. Similarmente, tanto el porcentaje de infectados cuya gravedad requiere hospitalización como el porcentaje que termina falleciendo es más alto entre los adultos mayores \cite{Verity2020}. El nivel socioeconómico tampoco puede ser ignorado; el menor acceso a la salud \cite{Wang2020}, sumado al desempleo y la existencia de enfermedades crónicas previas \cite{Ahmed2020} son un agravante que hace más vulnerables a los niveles socioeconómicos más pobres.

Los datos de movilidad han sido utilizados ampliamente para modelar el avance de la enfermedad \cite{Lai2020}\cite{Kraemer2020}\cite{Chinazzi2020}; en la mayoría de los países la movilidad explica una parte importante de la variaciones en la transmisibilidad \cite{Nouvellet2021}. \cite{Chang2021} nota como las diferencias en movilidad explican las diferencias en transmisión en diferentes grupos económicos, notando que además las clases de menor nivel socioeconómico se enfrentan a ambientes más riesgosos.

% El impacto de la enfermedad ha sido heterogéneo entre la población. Se han señalado al factor socioeconómico \cite{Ahmed2020} como un agravante... y son conocidos los efectos de la enfermedad en los adultos mayores.

% Los datos de movilidad han sido utilizados ampliamente para modelar el avance de la enfermedad \cite{Lai2020}\cite{Oliver2020}, se correlacionan bien con los contactos \cite{Prem2020} y la transmisibilidad \cite{Nasan2021} (ver bien Nasan). \cite{Chang2021} nota como las diferencias en movilidad explican las diferencias en transmisión en diferentes grupos económicos, notando que además las clases de menor nivel socioeconómico se enfrentan a ambientes más riesgosos.


\subsection{El brote de COVID-19 en Santiago de Chile }

La Región Metropolitana (RM) contiene a la ciudad de Santiago, la capital de Chile y su centro administrativo. Está formada por 52 comunas, las cuales varían significativamente en términos socioeconómicos. Su población es de unos \(7.3\) millones de personas, un 40\% aproximadamente de la población del país.

%[\url{https://www.gob.cl/coronavirus/gestionpandemia/}]
Las primeras medidas tomadas por el Gobierno de Chile para enfrentar la pandemia se centraron en fortalecer la capacidad hospitalaria: aumento en la cantidad de camas UCI, integración de los sistemas de salud público y privado, fortalecimiento de la capacidad de testeo y trazabilidad, preparación de residencias sanitarias, importación de material y equipamiento sanitario.

Se implementaron además \cite{Tariq2021a} medidas de distanciamiento social como la prohibición de eventos de gran concurrencia y el cierre de instituciones educativas. Se llevaron a cabo cuarentenas localizadas, hasta entrar en cuarentena total a mediados de mayo de 2020. En julio, una vez superado el primer \textit{peak} de casos, se comienza a aplicar el Plan Paso a Paso, una estrategia de desconfinamiento  gradual que sigue operando con modificaciones hasta la fecha.

Ya a fines de mayo de 2020, \cite{Olivares2020} había hecho notar "la dificultad de implementar cuarentenas obligatorias en comunas de menores recursos, donde la gente vive al día y genera sus ingresos con su trabajo diario", sugiriendo "que al aplicar las cuarentenas en comunas de menores niveles socioeconómicos, el gobierno las complemente con herramientas que apoyen a sus habitantes y provea de recursos para cubrir necesidades urgentes". Algunas medidas tomadas en esta dirección fueron la entrega de bonos e ingresos suplementarios, beneficios para el pago de servicios básicos, canastas de alimentos, etc.

Sin embargo, en \cite{Bennett2021} se muestra que, si bien las cuarentenas fueron efectivas en las comunas de mayores ingresos, no parecen tener un efecto significativo en las comunas de menores ingresos. Atribuye esto en parte a las diferencias en el cumplimiento de las medidas en términos de movilidad, así como a las diferentes capacidades de testeo. Estos resultados son similares a los obtenidos en \cite{Gozzi2021}. En \cite{Mena2021} se explora aún más este tópico, encontrando además una fuerte correlación entre el nivel socioeconómico y la mortalidad, que afecta aún más a los menores de 40 años.

El 3 de febrero de 2021 comienza el proceso de vacunación masiva de la población \cite{MINSAL2021}, el cual se realiza en su mayor parte con las vacunas Sinovac, Pfizer-BioNTech y Oxford-AstraZeneca. Hacia marzo de 2022 se está aplicando la cuarta dosis de refuerzo; Chile se encuentra entre los países con mayor porcentaje de personas vacunadas, rondando el 90\% de la población inoculada con dos dosis, similar a países como Canadá y Singapur \cite{Mathieu2021}.

El Ministerio de Salud ha liberado varios conjuntos de datos, que incluyen series de tiempo de casos confirmados, UCI, fallecidos, varias de ellas agregadas por edad o comuna. Están disponibles en el GitHub del Ministerio de Ciencia \cite{MINCIENCIA}. 
