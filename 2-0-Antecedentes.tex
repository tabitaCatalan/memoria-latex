\section{Antecedentes: El caso del COVID-19} \label{sec:antecedentes}

\textit{Aquí voy a usar wikipedia...}

Puesto que la principal aplicación que desarrollaremos está relacionada al COVID-19, es necesario considerar algunas de sus características. El COVID-19 es una enfermedad causada por el coronavirus SARS-CoV-2. Sus síntomas son similares a una gripe; fiebre, tos, dificultad para respirar.
\begin{itemize}
\item Formas de contagio 
\item Casos sintomáticos y asíntomáticos 
%\item supercontagiadores 
\item Diferencias de impacto basadas en edad (niños asintomáticos, adultos mayores con tendencia a ser graves).

\end{itemize}


\subsection{El brote de COVID-19 en Santiago de Chile }

La ciudad de Santiago es la capital de Chile y su centro administrativo. Está formada por [x] comunas, las cuales varían significativamente en términos socioeconómicos. Su población es de [aprox 7] millones de personas, un [aprox 36\%] de la población del país, y sin embargo, ha reportado un [buscar porcentaje] de los casos de COVID-19.

El Ministerio de Salud ha liberado varios conjuntos de datos, que incluyen series de tiempo de casos confirmados, UCI, fallecidos, varias de ellas agregadas por edad o comuna. Están disponibles en el GitHub del Minsal \url{https://github.com/MinCiencia/Datos-COVID19}.



[\url{https://www.gob.cl/coronavirus/gestionpandemia/}]
Las primeras medidas para enfrentar la pandemia se centraron en fortalecer la capacidad hospitalaria: aumento en la cantidad de camas UCI, integración de los sistemas de salud público y privado, fortalecimiento de la capacidad de testeo y trazabilidad, preparación de residencias sanitarias, importación de material y equipamiento sanitario.

Se implementaron además medidas de distanciamiento social como la prohibición de eventos de gran concurrencia y el cierre de instituciones educativas. Se llevaron a cabo cuarentenas localizadas, hasta entrar en cuarentena total a mediados de mayo de 2020. En julio, una vez superado el primer \textit{peak} de casos, se comienza a aplicar el Plan Paso a Paso, una estrategia de desconfinamiento  gradual que sigue operando hasta la fecha.

Ya a fines de mayo de 2020, \cite{Olivares2020} había hecho notar "la aprensión sobre la dificultad de implementar cuarentenas obligatorias en comunas de menores recursos, donde la gente vive al día y genera sus ingresos con su trabajo diario", sugiriendo "que al aplicar las cuarentenas en comunas de menores niveles socioeconómicos, el gobierno las complemente con herramientas que apoyen a sus habitantes y provea de recursos para cubrir necesidades urgentes". Algunas medidas tomadas en esta dirección fueron la entrega de bonos e ingresos suplementarios, beneficios para el pago de servicios básicos, canastas de alimentos, etc.

Sin embargo (septiembre 2020), en \cite{Bennett2021} se muestra que, si bien las cuarentenas fueron efectivas en las comunas de mayores ingresos, no parecen tener un efecto significativo en las comunas de menores ingresos. Atribuye esto en parte a las diferencias en el cumplimiento de las medidas en términos de movilidad, así como a las diferentes capacidades de testeo. Estos resultados son similares a los obtenidos en \cite{Gozzi2021}. En \cite{Mena2021} se explora aún más este tópico, encontrando además una fuerte correlación entre el nivel socioeconómico y la mortalidad, que afecta aún más a los menores de 40 años.


%Se trabajaron acuerdos con varios laboratorios, y se comenzó a vacunar a la población. Se comienza a vacunar masivamente a la población a comienzos de febrero de 2021.

%Mediados de febrero; plan de volver a clases.
