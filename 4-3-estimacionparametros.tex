\section{Estimación de parámetros}

%\subsection{Modelo con una clase}
\subsection{Modelo multiclase con datos sintéticos}

Usando condiciones iniciales, un valor de P... matrices de convarianza... 

\begin{figure}[h]
\centering
\includegraphics[width=0.99\textwidth]{img/resultados/synth/kalman_grouped_allstates_allgroups\parameterstring}
\caption{Resultados obtenidos con Filtro de Kalman Suavizado para el caso sintético.}
\label{synth-all-nohigh}
\end{figure}



\begin{figure}
     \centering
     \begin{subfigure}[b]{\textwidth}
         \centering
         \includegraphics[width=.8\textwidth]{img/resultados/synth/kalman_grouped_E_high1\parameterstring}
         \caption{Clase \(1\).}
     \end{subfigure}
     \hfill
     \begin{subfigure}[b]{\textwidth}
         \centering
         \includegraphics[width=0.8\textwidth]{img/resultados/synth/kalman_grouped_E_high2\parameterstring}
         \caption{Clase \(2\).}
     \end{subfigure}
        \caption{Casos Expuestos, comparando resultados obtenidos con solución real, en valores absolutos y normalizados con respecto a la cantidad de personas por clase.}
        \label{synth-e-comp-high}
\end{figure}

\begin{figure}
     \centering
     \begin{subfigure}[b]{\textwidth}
         \centering
         \includegraphics[width=.99\textwidth]{img/resultados/synth/kalman_grouped_alpha_high1\parameterstring}
         \caption{Clase \(1\).}
     \end{subfigure}
     \hfill
     \begin{subfigure}[b]{\textwidth}
         \centering
         \includegraphics[width=0.99\textwidth]{img/resultados/synth/kalman_grouped_alpha_high2\parameterstring}
         \caption{Clase \(2\).}
     \end{subfigure}
        \caption[Factor sanitario, caso sintético]{Factor sanitario, comparando resultados obtenidos con función de control usada para general los datos, acotando el dominio en el eje \(y\) para mejor apreciación.}
        \label{synth-alpha-comp-high}
\end{figure}


\subsection{Análisis de sensibilidad}
% A los ambientes 
% A los riesgos de contagio de sintomatico y asintomatico
\subsection{Modelo con datos reales}


\begin{figure}[h]
\centering
\includegraphics[width=0.99\textwidth]{img/resultados/kalman_grouped_allstates_allgroups\parameterstring}
\caption{Resultados obtenidos con Filtro de Kalman Suavizado para el caso real.}
\label{all-nohigh}
\end{figure}



\begin{figure}
     \centering
     \begin{subfigure}[b]{0.47\textwidth}
         \centering
         \includegraphics[width=\textwidth]{img/resultados/kalman_grouped_I_high1\parameterstring}
         \caption{Clase \(1\).}
     \end{subfigure}
     \hfill
     \begin{subfigure}[b]{.47\textwidth}
         \centering
         \includegraphics[width=\textwidth]{img/resultados/kalman_grouped_I_high2\parameterstring}
         \caption{Clase \(2\).}
     \end{subfigure}
     \hfill
     \begin{subfigure}[b]{.47\textwidth}
         \centering
         \includegraphics[width=\textwidth]{img/resultados/kalman_grouped_I_high3\parameterstring}
         \caption{Clase \(3\).}
     \end{subfigure}
     \hfill
     \begin{subfigure}[b]{.47\textwidth}
         \centering
         \includegraphics[width=\textwidth]{img/resultados/kalman_grouped_I_high4\parameterstring}
         \caption{Clase \(4\).}
     \end{subfigure}
     \hfill
     \begin{subfigure}[b]{.47\textwidth}
         \centering
         \includegraphics[width=\textwidth]{img/resultados/kalman_grouped_I_high5\parameterstring}
         \caption{Clase \(5\).}
     \end{subfigure}
     \hfill
     \begin{subfigure}[b]{.47\textwidth}
         \centering
         \includegraphics[width=\textwidth]{img/resultados/kalman_grouped_I_allclass\parameterstring}
         \caption{Todas las clases.}
     \end{subfigure}
        \caption{Cantidad de Infectados estimados a partir de datos reales, en valores absolutos y normalizados con respecto a la cantidad de personas por clase.}
        \label{e-comp-high}
\end{figure}



\begin{figure}
     \centering
     \begin{subfigure}[b]{0.47\textwidth}
         \centering
         \includegraphics[width=\textwidth]{img/resultados/kalman_grouped_S_high1\parameterstring}
         \caption{Clase \(1\).}
     \end{subfigure}
     \hfill
     \begin{subfigure}[b]{.47\textwidth}
         \centering
         \includegraphics[width=\textwidth]{img/resultados/kalman_grouped_S_high2\parameterstring}
         \caption{Clase \(2\).}
     \end{subfigure}
     \hfill
     \begin{subfigure}[b]{.47\textwidth}
         \centering
         \includegraphics[width=\textwidth]{img/resultados/kalman_grouped_S_high3\parameterstring}
         \caption{Clase \(3\).}
     \end{subfigure}
     \hfill
     \begin{subfigure}[b]{.47\textwidth}
         \centering
         \includegraphics[width=\textwidth]{img/resultados/kalman_grouped_S_high4\parameterstring}
         \caption{Clase \(4\).}
     \end{subfigure}
     \hfill
     \begin{subfigure}[b]{.47\textwidth}
         \centering
         \includegraphics[width=\textwidth]{img/resultados/kalman_grouped_S_high5\parameterstring}
         \caption{Clase \(5\).}
     \end{subfigure}
     \hfill
     \begin{subfigure}[b]{.47\textwidth}
         \centering
         \includegraphics[width=\textwidth]{img/resultados/kalman_grouped_S_allclass\parameterstring}
         \caption{Todas las clases.}
     \end{subfigure}
        \caption{Cantidad de Susceptibles estimados a partir de datos reales, en valores absolutos y normalizados con respecto a la cantidad de personas por clase.}
        \label{s-comp-high}
\end{figure}




\begin{figure}
     \centering
     \begin{subfigure}[b]{.99\textwidth}
         \centering
         \includegraphics[width=\textwidth]{img/resultados/kalman_grouped_alpha_allclass\parameterstring}
         \caption{Todas las clases.}
     \end{subfigure}
     \hfill
     \begin{subfigure}[b]{0.47\textwidth}
         \centering
         \includegraphics[width=\textwidth]{img/resultados/kalman_grouped_alpha_high1\parameterstring}
         \caption{Clase \(1\).}
     \end{subfigure}
     \hfill
     \begin{subfigure}[b]{.47\textwidth}
         \centering
         \includegraphics[width=\textwidth]{img/resultados/kalman_grouped_alpha_high2\parameterstring}
         \caption{Clase \(2\).}
     \end{subfigure}
     \hfill
     \begin{subfigure}[b]{.47\textwidth}
         \centering
         \includegraphics[width=\textwidth]{img/resultados/kalman_grouped_alpha_high3\parameterstring}
         \caption{Clase \(3\).}
     \end{subfigure}
     \hfill
     \begin{subfigure}[b]{.47\textwidth}
         \centering
         \includegraphics[width=\textwidth]{img/resultados/kalman_grouped_alpha_high4\parameterstring}
         \caption{Clase \(4\).}
     \end{subfigure}
     \hfill
     \begin{subfigure}[b]{.47\textwidth}
         \centering
         \includegraphics[width=\textwidth]{img/resultados/kalman_grouped_alpha_high5\parameterstring}
         \caption{Clase \(5\).}
     \end{subfigure}
        \caption[Factor sanitario estimado a partir de datos reales.]{Factor sanitario estimado a partir de datos reales. Las líneas grises corresponden a algunas fechas relevantes: (1) 13/may/2020 - Comienzo de la cuarentena total en la RM; (2) 25/oct/2020 - Plesbicito por la nueva constitución; (3) 4/ene/2021 - Comienza a funcionar el pase de vacaciones; (4) 1/feb/2021 - Comienza un plan de vacunación más intensivo (ver visualizador CMM); (5) 26/may/2021 - Un 50\% de la población en la RM tiene la primera dosis de la vacuna.}
        \label{s-comp-high}
\end{figure}


