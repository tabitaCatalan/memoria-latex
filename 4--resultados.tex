\chapter{Resultados} \label{chap:results}

%%%%%%%%%%%%%%%%%%%%%%%%%%%%%%%%%%%%%%%%%%%%%%%%%%%%%%%%%%%%%%%%%
% Índice 
%%%%%%%%%%%%%%%%%%%%%%%%%%%%%%%%%%%%%%%%%%%%%%%%%%%%%%%%%%%%%%%%% Resultados para cada uno de los objetivos específicos 

En este capítulo se exponen los resultados obtenidos con la metodología propuesta. La sección \ref{res:matriz} está dedicada a la matriz de tiempos de residencia. Esta se estimó en dos etapas; en primer lugar, usando la Encuesta Origen-Destino tratada mediante el algoritmo \ref{alg:timematrix} para obtener el comportamiento de los santiaguinos en condiciones normales. A continuación, se utilizan datos de movilidad para actualizar esta matriz, de forma que refleje las variaciones en pandemia.

En la sección \ref{sec:estimacion-results} se muestra la estimación del factor sanitario. Esto se hace en dos etapas; en primer lugar se calcula para un caso sintético, donde la solución buscada está disponible. El conjunto de parámetros que permite una buena recuperación del factor sanitario se extrapola al caso real, donde se utilizan los casos confirmados obtenidos de Datos-COVID19 presentados en la subsección \ref{sec:datos-minsal}. Se estudia además la sensibilidad de las estimaciones ante distintos parámetros.

Finalmente, en la sección \ref{sec:evalmodel-casoshipot}, se usan los valores estimados en la sección anterior para simular varios casos hipotéticos; ¿cómo habría sido el desarrollo de la pandemia si todos se cuidaran de la misma forma pero además pudieran guardar una cuarentena más estricta? ¿y si no se hacía cuarentena pero todos respetaban las medidas sanitarias de una mejor manera?.  Se estudian dos escenarios dados por el parámetro \(\beta_{\text{exterior}}\).
%En segundo lugar, en la subsección \ref{subsec:rt}, se calcula el valor del \(\mathcal{R}_t\) para Santiago, y se compara con valor calculado con el método de Cori \cite{Cori2013} por el Centro de Modelamiento Matemático.



% Resultados de la matriz con tiempo constantes 
\section{Matriz de tiempos de residencia} \label{res:matriz}

En esta sección se muestra la matriz de tiempos de residencia que se usará en las simulaciones. En primer lugar se mostrarán los resultados obtenidos a partir del algoritmo \ref{alg:timematrix}, los que permitirán tener una idea del comportamiento de las personas de la ciudad de Santiago en condiciones normales.

Primero se estudiarán los resultados obtenidos con una lista de ambientes extensiva, que incluye prácticamente todos los propósitos listados por la Encuesta Origen Destino. Con respecto a las clases, se utiliza una clasificación por edad, sexo y nivel dde ingresos. Esto entrega una matriz de 18 clases y 13 ambientes que refleja el comportamiento pre-pandeamia de los santiaguinos.

Como se discutió en \ref{met:decisiones}, esta matriz es demasiado granular y no hay datos suficientes para actualizarla con el comportamiento durante la pandemia. Por tanto, en segundo lugar se mostrará la matriz definitiva a utilizar en las simulaciones, la cual solo tiene 5 clases y 2 ambientes.

En segundo lugar, se mostrarán los datos de movilidad para los distintos grupos, y la matriz a tiempo variable. Esto nos dará una idea del comportamiento a lo largo de la pandemia. 

\subsection{Santiago en condiciones normales}

% El código para esta sección esta en el github...
El código para esta sección se encuentra en el repositorio de GitHub \textit{tabitaCatalan/lagrangian-time}, específicamente en el directorio \texttt{src/time\_residence\_matrix/}.\\


\noindent \textbf{Matriz detallada}\\
% Elección de clases y ambientes, lista de ambientes disponibles en la EOD, cómo se eligieron las clases, etc. Esto debería estar docuementado en el repo. 
% \ref{alg:timematrix} 
Para evaluar los resultados del algoritmo, se aplica a la Encuesta Origen Destino Santiago 2012, ya descrita en \ref{}. Las clases son elegidas combinando los tres criterios de la tabla \ref{table:clases-full}. El nivel socioeconómico es calculado en base a los datos de ingreso de cada persona provistos por la encuesta, agrupando a nivel de hogar y dividiendo por la cantidad de habitantes del hogar. Los ingresos de corte son elegidos de tal forma que cada tramo socioeconómico tiene unas \(20\,000\) personas.

\begin{table}[h!]
\centering
\begin{tabular}{||r|c||c||r|c||} 
 \hline
 \multicolumn{2}{||c||}{Edad (años)} & Sexo      & \multicolumn{2}{c||}{\begin{tabular}{@{}c@{}}Nivel Socioeconómico \\ (ingreso per cápita)\end{tabular}} \\
 \hline
 \textbf{Joven} & 0-24   & Hombre    & \textbf{Bajo}&\(\leq\) \$\(111\,666\)\\
 \textbf{Adulto} & 25-64 & Mujer     & \textbf{Medio}& \$ \(111\,667\) - \$ \(199\,999 \)\\
 \textbf{Mayor} & 65 o más &         & \textbf{Alto}&\(\geq\) \$ \(200\,000\)\\
 \hline
\end{tabular}
\caption{Criterios usados para obtener las clases de la matriz detallada a partir de la EOD2012 Santiago.}
\label{table:clases-full}
\end{table}


Los ambientes utilizados están basados en la lista de propósitos de viajes y los modos de transporte usados. Son los siguientes: \texttt{hogar}, \texttt{trabajo}, \texttt{estudios}, \texttt{compras}, \texttt{visitas}, \texttt{salud}, \texttt{trámites}, \texttt{recreación}, \texttt{transporte público}, \texttt{auto}, \texttt{caminata}, \texttt{bicicleta} y \texttt{otros}. La lista de propósitos de viajes y sus ambientes asociados se encuentra en la tabla \ref{table:ambientes-prop-full}. La lista de modos está en la tabla \ref{table:ambientes-modo-full}.

% \begin{table}[h!]
% \centering
% \begin{tabular}{||l|c||l|c||} 
%  \hline
%  \multicolumn{2}{||c||}{Propósito de viaje} &  \multicolumn{2}{c||}{Modo de transporte} \\
%  \hline
% 1. Al trabajo               & \texttt{trabajo}      & 1:Auto                        & \texttt{auto}\\
% 2. Por trabajo              & \texttt{trabajo}      & 2:Bus TS                      & \texttt{transporte público}\\
% 3. Al estudio               & \texttt{estudios}     & 3:Bus no TS                   & \texttt{transporte público}\\
% 4. Por estudio              & \texttt{estudios}     & 4:Metro                       & \texttt{transporte público}\\
% 5. De salud                 & \texttt{salud}        & 5:Taxi Colectivo              & \texttt{transporte público}\\
% 6. Visitar a alguien        & \texttt{visitas}      & 6:Taxi                        & \texttt{auto}\\
% 7. Volver a casa            & \texttt{hogar}        & 7:Bus TS - Bus no TS          & \texttt{transporte público}\\
% 8. Buscar o dejar a alguien & \texttt{visitas}      & 8:Auto - Metro                & \texttt{transporte público}\\
% 9. Comer o tomar algo       & \texttt{compras}      & 9:Bus TS - Metro              & \texttt{transporte público}\\
% 10.Buscar o dejar algo      & \texttt{compras}      & 10:Bus no TS - Metro          & \texttt{transporte público}\\
% 11.De compras               & \texttt{compras}      & 11:Taxi Colectivo - Metro     & \texttt{transporte público}\\
% 12.Tramites                 & \texttt{trámites}     & 12:Taxi - Metro               & \texttt{transporte público}\\
% 13.Recreación               & \texttt{recreación}   & 13:Otros - Metro              & \texttt{transporte público}\\
% 14.Otra actividad           & \texttt{otros}        & 14:Otros - Bus TS             & \texttt{transporte público}\\
%                             &                       & 15:Otros - Bus TS - Metro     & \texttt{transporte público}\\
%                             &                       & 16:Otros                      & \texttt{otros}\\
%                             &                       & 17:Caminata                   & \texttt{caminata}\\
%                             &                       & 18:Bicicleta                  & \texttt{bicicleta}\\
%  \hline
% \end{tabular}
% \caption{Propósitos de viaje y modos de transporte de la EOD2012, y sus ambientes asociados.}
% \label{table:ambientes-full}
% \end{table}

\begin{table}[h!]
\centering
\begin{tabular}{||l|c||} 
 \hline
 \multicolumn{2}{||c||}{Propósito de viaje} \\
 \hline
1. Al trabajo               & \texttt{trabajo}      \\
2. Por trabajo              & \texttt{trabajo}      \\
3. Al estudio               & \texttt{estudios}     \\
4. Por estudio              & \texttt{estudios}     \\
5. De salud                 & \texttt{salud}        \\
6. Visitar a alguien        & \texttt{visitas}      \\
7. Volver a casa            & \texttt{hogar}        \\
8. Buscar o dejar a alguien & \texttt{visitas}      \\
9. Comer o tomar algo       & \texttt{compras}      \\
10.Buscar o dejar algo      & \texttt{compras}      \\
11.De compras               & \texttt{compras}      \\
12.Tramites                 & \texttt{trámites}     \\
13.Recreación               & \texttt{recreación}   \\
14.Otra actividad           & \texttt{otros}        \\
 \hline
\end{tabular}
\caption{Propósitos de viaje de la EOD2012 y sus ambientes asociados.}
\label{table:ambientes-prop-full}
\end{table}

\begin{table}[h!]
\centering
\begin{tabular}{||l|c||} 
 \hline
 \multicolumn{2}{||c||}{Modo de transporte} \\
 \hline
 1:Auto                        & \texttt{auto}\\
 2:Bus TS                      & \texttt{transporte público}\\
 3:Bus no TS                   & \texttt{transporte público}\\
 4:Metro                       & \texttt{transporte público}\\
 5:Taxi Colectivo              & \texttt{transporte público}\\
 6:Taxi                        & \texttt{auto}\\
 7:Bus TS - Bus no TS          & \texttt{transporte público}\\
 8:Auto - Metro                & \texttt{transporte público}\\
 9:Bus TS - Metro              & \texttt{transporte público}\\
 10:Bus no TS - Metro          & \texttt{transporte público}\\
 11:Taxi Colectivo - Metro     & \texttt{transporte público}\\
 12:Taxi - Metro               & \texttt{transporte público}\\
 13:Otros - Metro              & \texttt{transporte público}\\
 14:Otros - Bus TS             & \texttt{transporte público}\\
 15:Otros - Bus TS - Metro     & \texttt{transporte público}\\
 16:Otros                      & \texttt{otros}\\
 17:Caminata                   & \texttt{caminata}\\
 18:Bicicleta                  & \texttt{bicicleta}\\
 \hline
\end{tabular}
\caption{Modos de transporte de la EOD2012 y sus ambientes asociados.}
\label{table:ambientes-modo-full}
\end{table}

La matriz obtenida \ref{img:Pmatrix-full} refleja el comportamiento esperado de la población de Santiago. Como era de esperarse, se pasa la mayor parte del tiempo en el hogar (de hecho, se redujo el tiempo en el hogar en 7 horas de tiempo de sueño). 

Algunas observaciones: los adultos pasan una cantidad considerable de tiempo en el trabajo, en especial los hombres. Los jóvenes pasan mucho tiempo en el ambiente estudios, especialmente los de clase alta; los jóvenes con menos ingresos pasan más tiempo en el trabajo. Los adultos mayores pasan mucho más tiempo en el hogar, pero los hombres pasan una cantidad importante de tiempo en el trabajo. 

El nivel socioeconómico medio pasa menos tiempo de compras y de visita que los demás. El auto es bastante más utilizado por la clase alta y por los adultos hombres de clase media. Los adultos mayores pasan más tiempo en el ambiente salud, especialmente los de clase baja.

En general los resultados son los esperados para una ciudad con una segregación importante como lo es Santiago. Es posible que en la última década se hayan reducido las diferencias de género, pero eso queda fuera del alcance de este trabajo.


\begin{figure}[!h]
\centering
\includegraphics[width=0.99\textwidth]{img/resultados/matrixP/matriz Pambientesyclases.pdf}
\caption[Matriz detallada de tiempos de residencia para Santiago]{Matriz detallada de tiempos de residencia para Santiago. Considera clases basadas en nivel socioeconómico (según ingreso promedio del hogar), edad y sexo. 13 ambientes, basados en los propósitos de viajes de la encuesta. Se restan 7 horas del tiempo en el hogar (tiempo de sueño) y se normalizan las filas para que sumen 1.}
\label{img:Pmatrix-full}
\end{figure}


\noindent \textbf{Matriz para las simulaciones}


\subsection{Santiago a lo largo del tiempo}

\section{Análisis de sensibilidad}


\begin{figure}
\centering
\begin{tikzpicture}
	\begin{pgfonlayer}{nodelayer}
		\node [style=none] (0) at (-2, 2) {};
		\node [style=none] (1) at (0, 2) {};
		\node [style=none] (2) at (-2, 1) {};
		\node [style=none] (3) at (0, 1) {};
		\node [style=none] (4) at (-2, 0) {};
		\node [style=none] (5) at (0, 0) {};
		\node [style=none] (6) at (-2, -1) {};
		\node [style=none] (7) at (0, -1) {};
		\node [style=none] (8) at (2, 2) {$0.15$};
		\node [style=none] (9) at (2, 1) {};
		\node [style=none] (10) at (2, 1) {};
		\node [style=none] (11) at (2, 1) {$0.25$};
		\node [style=none] (12) at (2, 0) {};
		\node [style=none] (13) at (2, 0) {$0.35$};
		\node [style=none] (14) at (2, -1) {};
		\node [style=none] (15) at (2, -1) {$0.45$};
		\node [style=none] (16) at (0, 3) {};
		\node [style=none] (17) at (0, 3) {Condición inicial $\alpha_0$};
		\node [style=none] (18) at (6, 3) {};
		\node [style=none] (19) at (6, 3) {$\beta_{\textrm{ext}}$};
		\node [style=none] (20) at (0, -3) {};
		\node [style=none] (21) at (0, -3) {Solución real};
		\node [style=none] (22) at (-2, -4) {};
		\node [style=none] (23) at (0, -4) {};
		\node [style=none] (24) at (2, -4) {};
		\node [style=none] (25) at (2, -4) {$1.88$};
		\node [style=none] (26) at (4, 2) {};
		\node [style=none] (27) at (6, 2) {};
		\node [style=none] (28) at (4, 1) {};
		\node [style=none] (29) at (6, 1) {};
		\node [style=none] (30) at (4, 0) {};
		\node [style=none] (31) at (6, 0) {};
		\node [style=none] (32) at (4, -1) {};
		\node [style=none] (33) at (6, -1) {};
		\node [style=none] (34) at (4, -2) {};
		\node [style=none] (35) at (6, -2) {};
		\node [style=none] (36) at (4, -3) {};
		\node [style=none] (37) at (6, -3) {};
		\node [style=none] (38) at (4, -4) {};
		\node [style=none] (39) at (6, -4) {};
		\node [style=none] (40) at (4, -5) {};
		\node [style=none] (41) at (6, -5) {};
		\node [style=none] (42) at (8, 2) {$1.2$};
		\node [style=none] (43) at (8, 1) {$1.5$};
		\node [style=none] (44) at (8, 0) {$1.8$};
		\node [style=none] (45) at (8, -1) {$2.1$};
		\node [style=none] (46) at (8, -2) {$2.4$};
		\node [style=none] (47) at (8, -3) {$2.7$};
		\node [style=none] (48) at (8, -4) {$3.0$};
		\node [style=none] (49) at (8, -5) {$3.3$};
	\end{pgfonlayer}
	\begin{pgfonlayer}{edgelayer}
		\draw [style={a0_1}] (0.center) to (1.center);
		\draw [style={a0_2}] (2.center) to (3.center);
		\draw [style={a0_3}] (4.center) to (5.center);
		\draw [style={a0_4}] (6.center) to (7.center);
		\draw [style=real] (22.center) to (23.center);
		\draw [style=beta1] (26.center) to (27.center);
		\draw [style=beta2] (28.center) to (29.center);
		\draw [style=beta3] (30.center) to (31.center);
		\draw [style=beta4] (32.center) to (33.center);
		\draw [style=beta5] (34.center) to (35.center);
		\draw [style=beta6] (36.center) to (37.center);
		\draw [style=beta7] (38.center) to (39.center);
		\draw [style=beta8] (40.center) to (41.center);
	\end{pgfonlayer}
\end{tikzpicture}

\caption{Leyenda sensibilidad ante \(\beta_{\text{exterior}}\).} \label{fig:legend-sensi-b}
\end{figure}



% Resultados de usar el filtro de kalman 
\section{Estimación de parámetros}

\subsection{Modelo con una clase}
\subsection{Modelo multiclase con datos sintéticos}

Usando condiciones iniciales, un valor de P... matrices de convarianza... 

\insertimage[\label{synth-all-nohigh}]{resultados/synth/kalman_grouped_allstates_allgroups\parameterstring}{width=0.99\textwidth}{Resultados obtenidos con Filtro de Kalman Suavizado para el caso sintético}

\begin{images}[\label{synth-e-comp-high}]{Casos Expuestos, comparando resultados obtenidos con solución real, en valores absolutos y normalizados con respecto a la cantidad de personas por clase.}
    \addimage{resultados/synth/kalman_grouped_E_high1\parameterstring}{width=0.8\textwidth}{Clase \(1\).}
    \addimage{resultados/synth/kalman_grouped_E_high2\parameterstring}{width=0.8\textwidth}{Clase \(2\).}
    %\addimage{ejemplos/test-image}{width=12cm}{Ciudad más grande}
\end{images}

\begin{images}[\label{synth-alpha-comp-high}]{Factor sanitario, comparando resultados obtenidos con función de control usada para general los datos, acotando el dominio en el eje \(y\) para mejor apreciación.}
    \addimage{resultados/synth/kalman_grouped_alpha_high1\parameterstring}{width=0.99\textwidth}{Clase \(1\).}
    \addimage{resultados/synth/kalman_grouped_alpha_high2\parameterstring}{width=0.99\textwidth}{Clase \(2\).}
    %\addimage{ejemplos/test-image}{width=12cm}{Ciudad más grande}
\end{images}


\subsection{Análisis de sensibilidad}
% A los ambientes 
% A los riesgos de contagio de sintomatico y asintomatico
\subsection{Modelo con datos reales}


\insertimage[\label{all-nohigh}]{resultados/kalman_grouped_allstates_allgroups\parameterstring}{width=0.99\textwidth}{Resultados obtenidos con Filtro de Kalman Suavizado para el caso sintético}

\begin{images}[\label{e-comp-high}]{Cantidad de Infectados estimados a partir de datos reales, en valores absolutos y normalizados con respecto a la cantidad de personas por clase.}
    \addimage{resultados/kalman_grouped_I_high1\parameterstring}{width=0.47\textwidth}{Clase \(1\).}
    \addimage{resultados/kalman_grouped_I_high2\parameterstring}{width=0.47\textwidth}{Clase \(2\).}
    \addimage{resultados/kalman_grouped_I_high3\parameterstring}{width=0.47\textwidth}{Clase \(3\).}
    \addimage{resultados/kalman_grouped_I_high4\parameterstring}{width=0.47\textwidth}{Clase \(4\).}
    \addimage{resultados/kalman_grouped_I_high5\parameterstring}{width=0.47\textwidth}{Clase \(5\).}
    \addimage{resultados/kalman_grouped_I_allclass\parameterstring}{width=0.47\textwidth}{Todas las clases.}
    %\addimage{ejemplos/test-image}{width=12cm}{Ciudad más grande}
\end{images}

\begin{images}[\label{s-comp-high}]{Cantidad de Susceptibles estimados a partir de datos reales, en valores absolutos y normalizados con respecto a la cantidad de personas por clase.}
    \addimage{resultados/kalman_grouped_S_high1\parameterstring}{width=0.47\textwidth}{Clase \(1\).}
    \addimage{resultados/kalman_grouped_S_high2\parameterstring}{width=0.47\textwidth}{Clase \(2\).}
    \addimage{resultados/kalman_grouped_S_high3\parameterstring}{width=0.47\textwidth}{Clase \(3\).}
    \addimage{resultados/kalman_grouped_S_high4\parameterstring}{width=0.47\textwidth}{Clase \(4\).}
    \addimage{resultados/kalman_grouped_S_high5\parameterstring}{width=0.47\textwidth}{Clase \(5\).}
    \addimage{resultados/kalman_grouped_S_allclass\parameterstring}{width=0.47\textwidth}{Todas las clases.}
    %\addimage{ejemplos/test-image}{width=12cm}{Ciudad más grande}
\end{images}

\begin{images}[\label{alpha-comp}]{Factor sanitario estimado a partir de datos reales. Las líneas grises corresponden a algunas fechas relevantes: (1) 13/may/2020 - Comienzo de la cuarentena total en la RM; (2) 25/oct/2020 - Plesbicito por la nueva constitución; (3) 4/ene/2021 - Comienza a funcionar el pase de vacaciones; (4) 1/feb/2021 - Comienza un plan de vacunación más intensivo (ver visualizador CMM); (5) 26/may/2021 - Un 50\% de la población en la RM tiene la primera dosis de la vacuna.}
    \addimage{resultados/kalman_grouped_alpha_allclass\parameterstring}{width=0.99\textwidth}{Todas las clases}
    \addimage{resultados/kalman_grouped_alpha_high1\parameterstring}{width=0.47\textwidth}{Clase \(1\).}
    \addimage{resultados/kalman_grouped_alpha_high2\parameterstring}{width=0.47\textwidth}{Clase \(2\).}
    \addimage{resultados/kalman_grouped_alpha_high3\parameterstring}{width=0.47\textwidth}{Clase \(3\).}
    \addimage{resultados/kalman_grouped_alpha_high4\parameterstring}{width=0.47\textwidth}{Clase \(4\).}
    \addimage{resultados/kalman_grouped_alpha_high5\parameterstring}{width=0.47\textwidth}{Clase \(5\).}
    %\addimage{ejemplos/test-image}{width=12cm}{Ciudad más grande}
\end{images}



% Agrupar para usar datos reales, calcular Rt y comparar con Cori
\section{Evaluación del modelo mediante casos hipotéticos}\label{sec:evalmodel-casoshipot}

% Descripción de los tipos de cuarentena y las medidas de cuidado 
Las combinaciones entre los distintos niveles de cuidado y cuarentena presentados en \ref{met:evaluacion-hipot} dan lugar a una serie de casos, los cuales son presentados en la tabla \ref{table:casos-hipoteticos}. Se resuelve el sistema para cada caso, comenzando desde las condiciones iniciales y los parámetros \(\gamma_E, \gamma_I, \beta_{\text{exterior}}\) estimados con \textit{RTS smoother} en la sección anterior.

\begin{table}[h!]
\centering
\begin{tabular}{||l| c c c||} 
 \hline
 & \textbf{Cuidado insuficiente} & \begin{tabular}{@{}c@{}}\textbf{Cuidado} \\ \textbf{normal}\end{tabular} & \begin{tabular}{@{}c@{}}\textbf{Cuidado} \\ \textbf{extra}\end{tabular}\\ 
 \hline
 \textbf{Sin cuarentena} & Caso 1 & Caso 2 & Caso 3 \\ 
 \textbf{Cuarentena normal} & Caso 4 & Caso 5 & Caso 6 \\
 \textbf{Cuarentena fuerte} & Caso 7 & Caso 8 & Caso 9 \\
 \hline
\end{tabular}
\caption{Casos hipotéticos para distintas combinaciones de cuarentena y cuidado.}
\label{table:casos-hipoteticos}
\end{table}

Las figuras \ref{img:all-hip-S-N} y \ref{img:all-hip-I-N} presentan la evolución de los susceptibles \(S_i/N_i\) e infectados \(S_i/N_i\) (normalizados por el total de personas en cada clase), para cada uno de los casos. Se utilizan los mismos límites para los ejes \(y\) para facilitar la comparación.

Como era de esperarse, las diferencias socioeconómicas disminuyen debido a que la mayoría de los casos están utilizando, o bien la misma matriz de tiempos de residencia, o bien los mismos factores sanitarios (o ambos).


\begin{table}[h!]
\centering
\begin{tabular}{||m{2.5cm}|m{4cm} m{4cm} m{4cm}||} 
 \hline
 & Cuidado & \begin{tabular}{@{}c@{}}\textbf{Cuidado} \\ \textbf{normal}\end{tabular} & \begin{tabular}{@{}c@{}}\textbf{Cuidado} \\ \textbf{extra}\end{tabular}\\ 
 \hline
 \textbf{Sin cuarentena} & \textit{Caso 1}: Se contagia \(\sim 90\%\) de la población en los primeros 6 meses, de manera uniforme entre las distintas clases. Luego la enfermedad desaparece. & \textit{Caso 2}: Se contagia \(\sim 90\%\) de la población en los primeros 6 meses, de manera uniforme entre las distintas clases. Luego la enfermedad desaparece. & \textit{Caso 3}: Se logra controlar el alza de casos, alcanzando el \textit{peak} de mediados de julio de 2020 y el de inicios de mayo de 2021 con menos de la mitad de casos, e incluso se evita el rebrote de junio de 2021. \\ 
 \textbf{Cuarentena normal} & \textit{Caso 4}: El primer \textit{peak} de julio de 2020 alcanza más del doble de casos y tiene una reducción lenta, de tal forma que el \textit{peak} de mayo de 2021 es apenas registrado como un alza ligera. & \textit{Caso 5}: Desarrollo real de la pandemia, tomado como referencia& \textit{Caso 6}: La enfermedad alcanza un pequeño \textit{peak} en julio de 2020 y luego se extingue.\\
 \textbf{Cuarentena fuerte} & \textit{Caso 7} El primer \textit{peak} es muy similar en fecha y número de infectados, pero con un alza sostenida en el número de casos a partir septiembre de 2020, llegando a un \textit{peak} de más del doble de infectados en julio de 2021. & \textit{Caso 8} Se observan dos \textit{peaks} de unas 5000 personas, similar al Caso 3 pero sin poder contener el rebrote de junio de 2021. & \textit{Caso 9} \\
 \hline
\end{tabular}
\caption{Observaciones para cada caso hipotético, escenario con \(\beta_{\text{exterior}} = 68.\) (el escenario exterior domina los contagios).}
\label{table:descripcion-casos-hipot}
\end{table}


Se puede ver como en los casos 1 y 2, que corresponden a casos sin cuarentena con cuidado insuficiente o cuidado normal respectivamente, se contagia la gran mayoría de la población dentro de los primeros 6 meses considerados. En el caso con cuidado normal, la clase más acomodada resulta tener una incidencia bastante menor que las demás.

Es interesante que la simulación muestra como el cuidado extra (casos 3, 6 y 9) es suficiente por sí mismo de controlar el alza de casos, incluso sin ningún tipo de cuarentena (caso 3). Con cuarentena (casos 6 y 9), la enfermedad desaparece rápidamente de entre la población en los primeros meses. Esto desde luego no considera importación de casos externos.

% Aquí pongo todos los resultados de casos hipotéticos que obtuve... cuarentenas fuertes, etc... 
\begin{figure}[h]
\centering
\includegraphics[width=0.99\textwidth]{img/resultados/allhipcases_S-N_commonylim0_1\parameterstring}
\caption{Estimación de susceptibles, normalizados por el total de cada grupo \(S_i/N_i\), para cada caso hipotético.}
\label{img:all-hip-S-N}
\end{figure}

\begin{figure}[h]
\centering
\includegraphics[width=0.99\textwidth]{img/resultados/allhipcases_I-N_commonylim0-015\parameterstring}
\caption{Estimación de infectados normalizados por el total de cada grupo \(I_i/N_i\), para cada caso hipotético.}
\label{img:all-hip-I-N}
\end{figure}

La figura \ref{img:hip-3478-I-comp} muestra cuatro casos interesantes, comparándolos con la situación normal. 




En estos tres casos, el impacto es mayor sobre la clase 5 (la más vulnerable). Esto puede atribuirse al hecho de que es esa la clase que presenta en general un mayor factor sanitario y movilidad.
 
 

% Fechas 
% important_dates = [
%    Date(2020, 5, 13), # cuarentena total en la RM
%    Date(2020, 10, 25), # plesbicito por nueva constitución
%    Date(2021, 1, 4), # comienza a regir el permiso de vacaciones
%    Date(2021, 2, 1), # comienza un plan de vacunación más fuerte (según datos)
%    Date(2021, 5, 26) # un 50% de la población de Santiago tiene la primera dosis
%]
% casos interesantes: 3, 4, 7, 8

\begin{figure}
     \centering
     \begin{subfigure}[b]{.47\textwidth}
         \centering
         \includegraphics[width=\textwidth]{img/resultados/comparecase_3withnormal_I_\parameterstring}
         \caption{Caso \(3\): Cuidado extra sin cuarentena.}
     \end{subfigure}
     \hfill
     \begin{subfigure}[b]{.47\textwidth}
         \centering
         \includegraphics[width=\textwidth]{img/resultados/comparecase_4withnormal_I_\parameterstring}
         \caption{Caso \(4\): Cuidado insuficiente y cuarentena normal.}
     \end{subfigure}
     \hfill
     \begin{subfigure}[b]{.47\textwidth}
         \centering
         \includegraphics[width=\textwidth]{img/resultados/comparecase_7withnormal_I_\parameterstring}
         \caption{Caso \(7\): Cuidado insuficiente y cuarentena fuerte.}
     \end{subfigure}
     \hfill
     \begin{subfigure}[b]{.47\textwidth}
         \centering
         \includegraphics[width=\textwidth]{img/resultados/comparecase_8withnormal_I_\parameterstring}
         \caption{Caso \(8\): Cuidado normal y cuarentena fuerte.}
     \end{subfigure}
        \caption[Personas infectadas para caso hipotéticos seleccionados.]{Personas infectadas para caso hipotéticos seleccionados, con la estimación de los casos reales de fondo. Diferentes límites para el eje \(y\). Las líneas grises corresponden a las fechas relevantes de la tabla \ref{table:fechas-relevantes}.}
        \label{img:hip-3478-I-comp}
\end{figure}



% Esto ya es solo una pregunta... pero se puede hacer el proceso a la inversar? obtener estimaciones de movilidad para distintos grupos usando los datos de infecciones.... el problema es que el mejor indicador son los fallecidooooos y los resultados dependen mucho de las tasas de muerte.
