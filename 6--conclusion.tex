\begin{conclusion} \label{chap:conclu}

% Responder la pregunta de investigación 
% Se cumplieron los objetivos? 
% Reflexionar en el trabajo realizado 
% Aportes del trabajo 
% Extensiones posibles 

El objetivo principal... 

Los objetivos específicos fueron logrados en su mayor parte, puesto que fue posible plantear una forma de estimar el factor sanitario y realizar una implementación para el caso de estudio. Sin embargo, era deseable poder incluir en el modelo ambientes como \texttt{trabajo}, \texttt{estudios}, \texttt{compras} o \texttt{transporte público}, o considerar grupos etarios, y esto no fue logrado. 

Con respecto al caso de estudio específicamente...

La utilidad del framework... resultados interesantes...

Las principales contribuciones de este trabajo. En primer lugar, la propuesta de una metodología que permite estudiar las medidas de cuidado de forma independiente a la movilidad. En segundo lugar, la aplicación del \textit{framework} lagrangiano de clases y ambientes presentado por \cite{Bichara2018} a un caso de estudio, lo cual no había sido hecho con anterioridad. Este trabajo guarda, sin embargo, cierta similitud con \cite{Shikhmurzaev}, que plantea un modelo que incorpora patrones de actividad.

El framework propuesto puede aplicarse a una amplia variedad de modelos, compartimientos, parámetros y observaciones. Algunas posibles extensiones son considerar los efectos de la vacunación en los compartimientos, usar comunas en lugar de grupos socioeconómicos, usar un filtro que se comporte mejor frente a no linealidades (como el filtro \textit{unscented} o ``sin olor''), ampliar el modelo para considerar fallecidos, etc.


En tercer lugar, la implementación en Matlab de la técnica propuesta por \cite{Munizaga2011} de obtención de uso del tiempo partir de los viajes de una Encuesta Origen-Destino. Esta implementación es de código abierto y se encuentra disponible en el repositorio \url{https://github.com/tabitaCatalan/lagrangian-time}. Se utilizan datos de la Encuesta Origen-Destino Santiago 2012.

En cuarto lugar, el desarrollo de una librería, escrita en el lenguaje Julia, para el trabajo con Filtro de Kalman con dinámica y observadores lineales y/o no lineales. Esta se encuentra documentada, es modular, extensible y de código abierto. Está disponible en el repositorio \url{https://github.com/tabitaCatalan/kalman}. Existen librerías similares en el mismo lenguaje como KalmanFilters.jl y Kalman.jl, pero estas no fueron usadas debido a que en la etapa de decidir qué Filtro utilizar, ninguna contaba con todas las opciones que se quería explorar. 

En quinto lugar, el planteamiento de una metodología para el ajuste de los hiperparámetros relacionados a la covarianza del filtro de Kalman, al estimar parámetros de un modelo epidemiológico. Este era un vacío en trabajos como \cite{Hasan2020} y \cite{Sameni2020}, donde se limitan a elegir un valor fijo para el caso particular que trabajan, sin ofrecer una justificación.

Finalmente, la aplicación del modelo al desarrollo de la pandemia de COVID-19 en la ciudad de Santiago de Chile, lo que permite una estimación de la vulnerabilidad/factor sanitario, de forma independiente a la movilidad. Modelos anteriores de Santiago como \cite{Gozzi2021} solo consideran movilidad.

Finalmente, el repositorio \url{https://github.com/tabitaCatalan/CovidMTK} contiene el código utilizado en el análisis del caso de estudio del COVID-19 en Santiago. Este repositorio hace uso de la novedosa librería ModelingToolkit.jl \cite{ma2021modelingtoolkit}, aun en una fase inicial de desarrollo, la cual, de forma similar a \textit{software} como Modelica o Simulink, ofrece herramientas para el trabajo con modelos de ecuaciones diferenciales, cálculo simbólico de derivadas, de forma eficiente, y extensible.

% Comparación con la literatura de la metodología 
% Se cumplieron los objetivos?? Era atingentes los objetivos? 


\end{conclusion}